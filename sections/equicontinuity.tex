\section{Equicontinuity}

\begin{definition}[Topological dynamical system]
	A \emph{topological dynamical system} (TDS) is a pair $(X, T)$, where
	\begin{itemize}
		\item $X$ is a non-empty, compact metric space,
		\item $T$ is a topological group,
		\item $\cdot : T \times  X \to X$ is a continuous group action, that is $\forall x \in X, \ s,t \in T$: 
		\begin{enumerate}
			\item $ e \cdot x = x$,
			\item $(st)\cdot x = s \cdot (t \cdot x)$.
		\end{enumerate}
	\end{itemize}
\end{definition}

\begin{definition}[Morphism of TDS]
	Let $(X, T)$ and $(Y,T)$ be topological dynamical systems. A map $\pi : X \to Y$ is called a \emph{homomorphism} from $(X, T) \to (Y, T)$ if $\pi$ is continuous and $\forall x \in X, t \in T$:
	\begin{equation*}
		t \pi (x) = \pi(tx).
	\end{equation*}
	It is an isomorphism if $\pi$ is a homeomorphism.
\end{definition}

\begin{definition}[Factor of TDS]
	Let $(X, T)$ and $(Y, T)$ be topological dynamical systems and $\pi: X \to Y$ a homomorphism. Then $\pi$ is called a \emph{factor map} if it is surjective.
	If there exists a factor map $X \to Y$ then $Y$ is called a \emph{factor} of $X$. If $\pi$ is an isomorphism then $(X, T)$ and $(Y, T)$ are \emph{conjugates}.
\end{definition}

\begin{proposition}[Quotient system]
	Let $(X, T)$ be a TDS. Let $R \subset X \times X$ be a $T$-invariant (i.e. $\forall t \in T, \forall (x, x^\prime ) \in R : (tx, tx^\prime) \in R$) and closed equivalence relation (ICER). Then the \emph{quotient} $(X/R, T)$ with $T$-action defined by $t[x] := [tx]$ is a TDS.
\end{proposition}
\begin{proof}
	Let $\pi: X \to X/R$ be the projection, which is continuous. The quotient $X/R$ is a Hausdorff space, because $X$ is a compact Hausdorff space and $R$ is closed (not trivial). The compactness of $X/R$ follows from $X/R = \pi(X)$ and the compactness of $X$ together with the continuity of $\pi$. The $T$-invariance of $R$ implies the well-definedness of the $T$-action. To show that the $T$ action is continuous: Define the equivalence relation
	\begin{equation*}
		R^\prime := \{ ((t_1,x_1), (t_2,x_2)) \in (T \times X) \times (T \times X): (x_1, x_2) \in R, t_1 = t_2\}.
	\end{equation*}
	Then the $T$-action induces a continuous map $\cdot : (T \times X)/R^\prime \to (T\times X)/R^\prime$ by the universal property of the quotient topology. The continuous map $(t, x) \mapsto (t, [x])$ induces the continuous and bijective map $(T\times X)/R^\prime \ni [(t, x)] \mapsto (t, [x])$. This map is closed\footnote{this is because $x \mapsto [x]$ is closed and proper + thm 5.17.5 of \url{https://stacks.math.columbia.edu/tag/005M}}, which implies that it is a homeomorphism.
\end{proof}

\begin{proposition}[Equivalence Factors and ICERs]
	Let $(X, T)$ be a TDS. If $R \subset X \times X$ is an ICER, then the quotient map $\pi_R: X \to X/R$ defines a factor of map.
	
	If $(Y, T)$ is a factor of $(X, T)$ with factor map $\pi: X \to Y$, then
	\begin{equation*}
		R_\pi := \{(x, y) \in X \times X;\ \pi(x) = \pi(y)\}
	\end{equation*}
	defines an ICER and the systems $(Y, T)$ and $(X/R_\pi, T)$ are conjugates.
	
	If $(Z, T)$ is an other factor of $(X, T)$ with factor map $\rho: X \to Y$ and corresponding ICER $R_\rho$. Then $(Y, T)$ and $(Z, T)$ are conjugates if and only if $R_\pi = R_\rho$. {\color{red} I know the $\Leftarrow$ directions is wrong.}
\end{proposition}
\begin{proof}
	 todo
\end{proof}

\begin{proposition}[Subsystem]
	Let $(X, T)$ be a TDS and $A \subseteq X$ a closed, $T$-invariant set.
	Then $(A, T)$ is also a TDS with the inherited $T$-action called the \emph{subsystem} of $(X, T)$ induced by $A$.
\end{proposition}
\begin{proof}
	Since $A$ is a closed subset of a compact Hausdorff space it is a compact Hausdorff space. The invariance of $A$ implies that the $T$ action is well defined, continuity is clear. 
\end{proof}

\begin{definition}[Product system]
	Let $(X_\alpha, T)_{\alpha \in A}$ be a family of topological dynamical systems. Then the \emph{product system} $(X, T)$ of the family is defined by
	\begin{equation*}
		X := \prod_{\alpha \in A} X_\alpha, \quad t (x_\alpha)_{\alpha \in A} := (t x_\alpha)_{\alpha \in A}.
	\end{equation*}
\end{definition}

\begin{definition}[Equicontinuous TDS]
	Let $(X,T)$ be a TDS. Then $(X,T)$ is called \emph{equicontinuous}, if
	\begin{equation*}
		\forall \varepsilon \in (0, \infty) \exists \delta \in (0, \infty): \forall x , x^\prime \in X :  d(x, x^\prime) < \delta  \Rightarrow \forall t \in T: d(tx, tx^\prime) < \varepsilon.
	\end{equation*}
\end{definition}

\begin{proposition}[Isomorphism invariance of equicontinuity]
	Let $(X, T)$ and $(Y, T)$ be two isomorphic topological dynamical systems. If $(X, T)$ is equicontinuous, so is $(Y, T)$.
\end{proposition}
\begin{proof}
	todo
\end{proof}

\begin{proposition}[Products of equicontinuous systems]
	Let $(X_\alpha, T, \cdot_\alpha)_{\alpha \in A}$ be a family of equicontinuous dynamical systems. Then the product $(\prod_{\alpha \in A} X_\alpha, T, \cdot)$ is also equicontinuous.
\end{proposition}
\begin{proof}
	todo
\end{proof}

\begin{proposition}[Subsystems of equicontinuous systems]
	Let $(X, T)$ be an equicontinuous dynamical systems and $A \subseteq X$ a closed $T$-invariant set. Then $(A, T)$ is equicontinuous.
\end{proposition}
\begin{proof}
	This is clear.
\end{proof}