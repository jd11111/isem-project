\section{Equicontinuity}

\begin{definition}[Topological dynamical system]
	A \emph{topological dynamical system} (TDS) is a pair $(X, T)$, where
	\begin{itemize}
		\item $X$ is a non-empty, compact metric space,
		\item $T$ is a topological group,
		\item $\cdot : T \times  X \to X$ is a continuous group action, that is $\forall x \in X, \ s,t \in T$: 
		\begin{enumerate}
			\item $ e \cdot x = x$,
			\item $(st)\cdot x = s \cdot (t \cdot x)$.
		\end{enumerate}
	\end{itemize}
\end{definition}

\begin{proposition}[Subsystem]
	Let $(X, T)$ be a TDS and $A \subseteq X$ a closed, $T$-invariant set.
	Then $(A, T)$ is also a TDS with the inherited $T$-action called the \emph{subsystem} of $(X, T)$ induced by $A$.
\end{proposition}
\begin{proof}
	Since $A$ is a closed subset of a compact Hausdorff space it is a compact Hausdorff space. The invariance of $A$ implies that the $T$ action is well defined, continuity is clear. 
\end{proof}

\begin{definition}[Product system]
	Let $(X_\alpha, T)_{\alpha \in A}$ be an at most countable family of topological dynamical systems. Then the \emph{product system} $(X, T)$ of the family is defined by
	\begin{equation*}
		X := \prod_{\alpha \in A} X_\alpha, \quad t (x_\alpha)_{\alpha \in A} := (t x_\alpha)_{\alpha \in A}.
	\end{equation*}
\end{definition}

\begin{definition}[Morphism of TDS]
	Let $(X, T)$ and $(Y,T)$ be topological dynamical systems. A map $\pi : X \to Y$ is called a \emph{homomorphism} from $(X, T) \to (Y, T)$ if $\pi$ is continuous and $\forall x \in X, t \in T$:
	\begin{equation*}
		t \pi (x) = \pi(tx).
	\end{equation*}
	It is an isomorphism if $\pi$ is a homeomorphism.
\end{definition}

\begin{definition}[Factor of TDS]
	Let $(X, T)$ and $(Y, T)$ be topological dynamical systems and $\pi: X \to Y$ a homomorphism. Then $\pi$ is called a \emph{factor map} if it is surjective.
	If there exists a factor map $X \to Y$ then $Y$ is called a \emph{factor} of $X$. If $\pi$ is an isomorphism then $(X, T)$ and $(Y, T)$ are \emph{conjugates}.
\end{definition}

\begin{proposition}[Quotient system]
	Let $(X, T)$ be a TDS. Let $R \subset X \times X$ be a $T$-invariant (i.e. $\forall t \in T, \forall (x, x^\prime ) \in R : (tx, tx^\prime) \in R$) and closed equivalence relation (ICER). Then the \emph{quotient} $(X/R, T)$ with $T$-action defined by $t[x] := [tx]$ is a TDS.
\end{proposition}
\begin{proof}
	Let $\pi_R: X \to X/R$ be the projection, which is continuous. The quotient $X/R$ is a Hausdorff space, because $X$ is a compact Hausdorff space and $R$ is closed (not trivial). The compactness of $X/R$ follows from $X/R = \pi(X)$ and the compactness of $X$ together with the continuity of $\pi$. The $T$-invariance of $R$ implies the well-definedness of the $T$-action. To show that the $T$ action is continuous: Define the equivalence relation
	\begin{equation*}
		R^\prime := \{ ((t_1,x_1), (t_2,x_2)) \in (T \times X) \times (T \times X): (x_1, x_2) \in R, t_1 = t_2\}.
	\end{equation*}
	Then the $T$-action induces a continuous map $\cdot : (T \times X)/R^\prime \to (T\times X)/R^\prime$ by the universal property of the quotient topology. The continuous map $(t, x) \mapsto (t, [x])$ induces the continuous and bijective map $(T\times X)/R^\prime \ni [(t, x)] \mapsto (t, [x])$. This map is closed\footnote{this is because $x \mapsto [x]$ is closed and proper + thm 5.17.5 of \url{https://stacks.math.columbia.edu/tag/005M}}, which implies that it is a homeomorphism.
\end{proof}

\begin{proposition}[Connection between Factors and ICERs]
	Let $(X, T)$ be a TDS. If $R \subset X \times X$ is an ICER, then the quotient map $\pi_R: X \to X/R$ defines a factor of map.
	
	If $(Y, T)$ is a factor of $(X, T)$ with factor map $\pi: X \to Y$, then
	\begin{equation*}
		R_\pi := \{(x, y) \in X \times X;\ \pi(x) = \pi(y)\}
	\end{equation*}
	defines an ICER and the systems $(Y, T)$ and $(X/R_\pi, T)$ are conjugated.
\end{proposition}
\begin{proof}
	The projection $\pi_R$ is continuous by definition of the quotient topology. For $x \in X$ and $t \in T$ we have
	\begin{align*}
		\pi_R(tx) = [tx] = t[x] = t\pi_R(x),
	\end{align*}
	so $\pi_R$ is indeed a factor map.
	
	It is clear that $R_\pi$ is closed, since $\pi$ is continuous. For $(x, y) \in R_\pi$ and $t \in T$ we have
	\begin{align*}
		\pi(tx) = t\pi(x) = t\pi(y) = \pi(ty),
 	\end{align*}
 	so $(tx, ty) \in R_\pi$, so $R_\pi$ is $T$-invariant. For $x \in X$ we have $\pi(x) = \pi(x)$ so $(x, x) \in R_\pi$, so $R_\pi$ is reflexive. Further for $(x, y) \in R_\pi$ we have $\pi(y) = \pi(x)$ so $(y, x) \in R_\pi$, which is therefore symmetric. Also for $x, y, z \in X$ with $(x, y) \in R_\pi$ and $(y, z) \i R_\pi$ we have $\pi(x) = \pi(y) = \pi(z)$, so $(x, z) \in R_\pi$ proving the transitivity of $R_\pi$. So we have shown that $R_\pi$ is an equivalence relation, hence an ICER.
 	
 	Define $\rho: X/R \to Y$ by $\rho([x]) = \pi(x)$. This is well defined, since $[x] = [y]$ implies $\pi(x) = \pi(y)$ by definition of $R_\pi$. For $U \subseteq Y$ open we have for all $x \in X$ that $\rho([x]) \in U$ if and only if $\pi(x) \in U$, so $[x] \in \rho^{-1}(U)$ if and only if $x \in \pi^{-1}(U)$. Thus $\pi_{R_\pi}^{-1}(\rho^{-1}(U)) = \pi^{-1}(U)$, which is open since $\pi$ is continuous. Thus by definition of the quotient topology $\rho^{-1}(U)$ is open, so $\rho$ is continuous. For $[x], [y] \in X/R_\pi$ with $\rho([x]) = \rho([y])$ we have $\pi(x) = \pi(y)$ so $[x] = [y]$, thus $\rho$ is injective. Also for $y \in Y$ there exists, since $\pi$ is surjective, a $x \in X$ such that $\pi(x) = y$. Therefore $\rho([x]) = \pi(x) = y$, proving that $\rho$ is surjective. Thus $\rho$ is bijective and since $X/R_\pi$ and $Y$ are compact Hausdorff $\rho$ is a homeomorphism. Lastly for $x \in X$ and $t \in T$
 	\begin{align*}
 		\rho(t[x]) = \rho([tx]) = \pi(tx) = t\pi(x) = t\rho([x])
 	\end{align*}
 	proves that $\rho$ is an isomorphism of TDSs.
\end{proof}

\begin{definition}[Equivalent and conjugated factors]
	Let $(X, T)$ be a TDS and factors $(Y, T)$ and $(Z, T)$ factors of it with factor maps $\pi: X \to Y$ and $\rho: X \to Z$.	Then the factors are called \emph(equivalent) if $R_\pi = R_\rho$ and \emph(conjugated) if $(Y, T)$ and $(Z, T)$ are conjugated.
	
	Similarly if $R$ and $S$ are two ICERs of $(X, T)$, they are called \emph{equivalent} if $R = S$ and \emph{conjugated} if $(X/R, T)$ and $(X/S, T)$ are conjugated.
\end{definition}

\begin{example}
	Let $(X, T)$ be a TDS. Let $\pi_i: X \times X \to X$ for $i = 1, 2$ be the projection on the $i$-th coordinate. Then $\pi_1$ and $\pi_2$ are factor maps of $(X \times X, T)$ to the factor $(X, T)$. They are clearly conjugates, as the factors are the same TDS. However it is easily verified that
	\begin{align*}
		R_{\pi_i} = \{((x_1, x_2), (y_1, y_2)) \in (X \times X)^2;\ x_i = y_i\}
	\end{align*}
	holds, so $R_{\pi_1} \neq R_{\pi_2}$, so the factors are not equivalent.
\end{example}

\begin{definition}[Equicontinuous TDS]
	Let $(X,T)$ be a TDS. Then $(X,T)$ is called \emph{equicontinuous}, if
	\begin{equation*}
		\forall \varepsilon > 0\ \exists \delta > 0: \forall x , y \in X :  d(x, y) < \delta  \Rightarrow \forall t \in T: d(tx, ty) < \varepsilon.
	\end{equation*}
\end{definition}

\begin{proposition}[Equicontinuity is independent on the choice of metric]
	Let $(X, T)$ be a TDS and $d, d'$ two metrics on $X$ that give rise to the same topology. Then $(X, T)$ is $d$-equicontinuous if and only if it is $d'$-equicontinuous.
\end{proposition}
\begin{proof}
	Fix $\varepsilon > 0$. The fact that $d$ and $d'$ give rise to the same topology, implies that $\mathrm{id}_X$ is $d$-$d'$ continuous. Since $X$ is compact it is even uniformly continuous, so there exists a $\tilde{\varepsilon} > 0$ such that
	\begin{align*}
		d(x, y) < \tilde{\varepsilon} \Rightarrow d'(x, y) < \varepsilon.
	\end{align*}
	Since $(X, T)$ is $d$-equicontinuous, there exists a $\tilde{\delta} > 0$ such that
	\begin{align*}
		d(x, y) < \tilde{\delta} \Rightarrow d(tx, ty) < \tilde{\varepsilon}
	\end{align*}
	holds for all $x, y \in X$ and $t \in T$. The identity map $\mathrm{id}_X$ is also $d'$-$d$ uniformly continuous, so there exists some $\delta > 0$ such that
	\begin{align*}
		d'(x, y) < \delta \Rightarrow d(x, y) < \tilde{\delta}
	\end{align*}
	holds for all $x, y \in X$. Thus we have
	\begin{align*}
		d'(x, y) < \delta \Rightarrow d(x, y) < \tilde{\delta} \Rightarrow d(tx, ty) < \tilde{\varepsilon} \Rightarrow d'(tx, ty) < \varepsilon
	\end{align*}
	for all $x, y \in X$ and $t \in T$. Hence $(X, T)$ is $d'$-equicontinuous.
\end{proof}

\begin{remark}
	The above proposition justifies, that we will often not give an explicit metric.
\end{remark}

\begin{proposition}[Isomorphism invariance of equicontinuity]
	Let $(X, T)$ and $(Y, T)$ be two isomorphic topological dynamical systems. If $(X, T)$ is equicontinuous, so is $(Y, T)$.
\end{proposition}
\begin{proof}
	todo
\end{proof}

\begin{proposition}[Products of equicontinuous systems]
	Let $(X_\alpha, T, \cdot_\alpha)_{\alpha \in A}$ be a family of equicontinuous dynamical systems. Then the product $(\prod_{\alpha \in A} X_\alpha, T, \cdot)$ is also equicontinuous.
\end{proposition}
\begin{proof}
	todo
\end{proof}

\begin{proposition}[Subsystems of equicontinuous systems]
	Let $(X, T)$ be an equicontinuous dynamical systems and $A \subseteq X$ a closed $T$-invariant set. Then $(A, T)$ is equicontinuous.
\end{proposition}
\begin{proof}
	This is clear.
\end{proof}