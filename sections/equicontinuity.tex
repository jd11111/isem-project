\section{Equicontinuity}

\begin{definition}[Topological dynamical system]
	A \emph{topological dynamical system} (TDS) is a triplet $(X,T,\cdot)$, where
	\begin{itemize}
		\item $X$ is a non-empty, compact metric space,
		\item $T$ is a topological group,
		\item $\cdot : T \times  X \to X$ is a continuous group action, that is $\forall x \in X, \ s,t \in T$: 
		\begin{enumerate}
			\item $ e \cdot x = x$,
			\item $(s \cdot t)\cdot x = s \cdot (t \cdot x)$.
		\end{enumerate}
	\end{itemize}
\end{definition}
\begin{remark}
	The map $\cdot$ is usually supressed in the notation.
\end{remark}

\begin{definition}[Morphism of TDS]
	Let $(X,T), \ (Y,T)$ be topological dynamical systems.
	A map $\pi : X \to Y$ is called a \emph{homomorphism} from $(X,T) \to (Y,T)$ if $\pi$ is continuous and $\forall x \in X, t \in T$:
	\begin{equation*}
		t \pi (x) = \pi (t x).
	\end{equation*}
\end{definition}

\begin{definition}[Factor of TDS]
	Let $(X,T), \ (Y,T)$ be topological dynamical systems and $\pi: X \to Y$ a homomorphism.
	Then $\pi$ is called a \emph{factor map} if it is surjective.
	If there exists a factor map $X \to Y$ then $Y$ is called a \emph{factor} of $X$.
\end{definition}

\begin{proposition}[Quotient system]
	Let $(X,T)$ be a TDS. Let $R \subset X \times X$ be a $T$-invariant\footnote{that is $\forall t \in T, \forall (x,x^\prime ) \in R : (xt ,x^\prime t) \in R$} and closed equivalence relation (ICER for short).
	Let $\pi : X \to X/R$ be the quotient map.
	Endow $X/R$ with the quotient topology, that is $U \subset X/R$ open $: \Leftrightarrow \pi^{-1}(U)$ is open.
	Define the $T$-action by $t \cdot [x] := [t \cdot x]$.
	Then $(X/R, T)$ is a TDS and $\pi$ is a factor map.
\end{proposition}
\begin{proof}
	The quotient $X/R$ is a Hausdorff space, because $X$ is a compact Hausdorff space and $R$ is closed (not trivial).
	The compactness of $X/R$ follows from $X/R = \pi (X)$ and the compactness of $X$ together with the continuity of $\pi$. The $T$-invariance of $R$ implies the well-definedness of the $T$-action and that $\pi$ is a factor map.
	To show that the $T$ action is continuous: Define the equivalence relation
	\begin{equation*}
		R^\prime := \{ ((t_1,x_1),(t_2,x_2)) \in (T \times X) \times (T \times X): (x_1, x_2) \in R, t_1 = t_2\}.
	\end{equation*}
	Then the $T$-action induces a continuous map $\cdot : (T \times X)/R^\prime \to (T\times X)/R^\prime$ by the universal property of the quotient topology.
	The continuous map $(t,x) \mapsto (t,[x])$ induces the continuous and bijective map $(T\times X)/R^\prime \ni [(t,x)] \mapsto ( t,[x])$. 
	This map is closed\footnote{this is because $x \mapsto [x]$ is closed and proper + thm 5.17.5 of \url{https://stacks.math.columbia.edu/tag/005M}}, which implies that it is a homeomorphism.
\end{proof}

\begin{definition}[Product system]
	Let $(X_\alpha,T, \cdot_\alpha)_{\alpha \in A}$ be a family of topological dynamical systems.
	Then the \emph{product system} $(X, T, \cdot)$ of the family is defined by
	\begin{equation*}
		X := \prod_{\alpha \in A} X_\alpha, \quad (t \cdot x )_{\alpha} := t \cdot_\alpha x.
	\end{equation*}
\end{definition}

\begin{definition}[Equicontinuous TDS]
	Let $(X,T)$ be a TDS.
	Then $(X,T)$ is called \emph{equicontinuous}, if
	\begin{equation*}
		\forall \varepsilon \in (0, \infty) \exists \delta \in (0, \infty): \forall x , x^\prime \in X :  d(x, x^\prime) <\delta  \Rightarrow \forall t \in T: d(tx,tx^\prime) < \varepsilon.
	\end{equation*}
\end{definition}

\begin{proposition}[Stability of equicontinuity]
	products of eq. cont. systems are eq. cont., subsystems of eq. cont. systems are eq. cont., equicontinuity is isomorphism invariant
\end{proposition}


\begin{proposition}[Subsystem]
	Let $(X,T)$ be a TDS and $A \subset X$ a closed, $T$-invariant set.
	Then $(A,T)$ is also a TDS with the inherited $T$-action called the \emph{subsystem} of $(X,T)$ induced by $A$.
\end{proposition}
\begin{proof}
	Since $A$ is a closed subset of a compact Hausdorff space it is a compact Hausdorff space. The invariance of $A$ implies that the $T$ action is well defined, continuity is clear. 
\end{proof}