\section{Preliminaries}
\begin{definition}[Topological dynamical system]
  A \emph{topological dynamical system} (TDS) is a triplet $(X,T,\cdot)$, where
  \begin{itemize}
    \item $X$ is a non-empty, compact topological Hausdorff space,
    \item $T$ is a topological group,
    \item $\cdot : X \times T \to X$ is a continuous group action, that is $\forall x \in X, \ s,t \in T$: 
    \begin{enumerate}
      \item $ x\cdot e = x$,
      \item $(x\cdot s)\cdot t = x \cdot (s \cdot t)$.
    \end{enumerate}
  \end{itemize}
\end{definition}
\begin{remark}
  The map $\cdot$ is usually supressed in the notation.
\end{remark}

\begin{definition}[Morphism of TDS]
  Let $(X,T), \ (Y,T)$ be topological dynamical systems.
  A map $\pi : X \to Y$ is called a \emph{homomorphism} from $(X,T) \to (Y,T)$ if $\pi$ is continuous and $\forall x \in X, t \in T$:
  \begin{equation*}
     \pi (x)   t = \pi ( x t).
  \end{equation*}
\end{definition}
\begin{definition}[Factor of TDS]
   Let $(X,T), \ (Y,T)$ be topological dynamical systems and $\pi: X \to Y$ a homomorphism.
   Then $\pi$ is called a \emph{factor map} if it is surjective.
   If there exists a factor map $X \to Y$ then $Y$ is called a \emph{factor} of $X$.
\end{definition}

\begin{proposition}[Quotient system]
  Let $(X,T)$ be a TDS. Let $R \subset X \times X$ be a $T$-invariant\footnote{that is $\forall t \in T, \forall (x,x^\prime ) \in R : (xt ,x^\prime t) \in R$} and closed equivalence relation (ICER for short).
  Let $\pi : X \to X/R$ be the quotient map.
  Endow $X/R$ with the quotient topology, that is $U \subset X/R$ open $: \Leftrightarrow \pi^{-1}(U)$ is open.
  Define the $T$-action by $[x] t := [x t]$.
  Then $(X/R, T)$ is a topological dynamical system with $X/R$ compact and $\pi$ is a factor map.
\end{proposition}
\begin{proof}
  The quotient $R/F$ is a Hausdorff space, because $X$ is a compact Hausdorff space and $R$ is closed (not trivial).
 The compactness of $R/F$ follows from $R/F = \pi (X)$ and the compactness of $X$ together with the continuity of $\pi$. The $T$-invariance of $R$ implies the well-definedness of the $T$-action and that $\pi$ is a factor map.
  \textbf{Warum ist die $T$-Wirkung stetig?!}
\end{proof}

\begin{definition}[Product system]
  Let $(X_\alpha,T, \cdot_\alpha)_{\alpha \in A}$ be a family of topological dynamical systems.
  Then the \emph{product system} $(X, T, \cdot)$ of the family is defined by
  \begin{equation*}
    X := \prod_{\alpha \in A} X_\alpha, \quad (x\cdot t)_{\alpha} := x \cdot_\alpha t.
  \end{equation*}
\end{definition}

\begin{definition}[Equicontinuous TDS]
  Let $(X,T)$ be a TDS.
  Then $(X,T)$ is called \emph{equicontinuous}, if for any $\alpha \in \mathcal{U}_X$ there is $\beta \in \mathcal{U}_X$ with $(x, x^\prime ) \in \beta \Rightarrow \forall t \in T: (xt,x^\prime t) \in \alpha$.
\end{definition}

\begin{proposition}
  products of eq. cont. flows are eq. cont., subflows are eq. cont.
\end{proposition}

\begin{proposition}[Subsystem]
  Let $(X,T)$ be a TDS and $A \subset X$ a closed, $T$-invariant set.
  Then $(A,T)$ is also a TDS with the inherited $T$-action called the \emph{subsystem} of $(X,T)$ induced by $A$.
\end{proposition}
\begin{proof}
  Since $A$ is a closed subset of a compact Hausdorff space it is a compact Hausdorff space. The invariance of $A$ implies that the $T$ action is well defined, continuity is clear. 
\end{proof}

\begin{theorem}
  Let $(X,T)$ be a TDS.
  Then there exists a smallest (with respect to set inclusion) ICER $R$ such that $(X/R, T)$ is equicontinuous.
    The system $(X/R,T)$ is called the \emph{maximal equicontinuous factor} of $(X,T)$.
\end{theorem}
\begin{proof}
  Let
  \begin{equation*}
    \mathcal{R} := \{ R \subset X \times X : R \ \text{ICER} \ \text{and} \ (X/R,T) \ \text{equicontinuous}\}.
  \end{equation*}
  Then $ X \times X \in \mathcal{R}$ and so the set is non empty.
  Now let 
  \begin{equation*}
    R:= \cap \mathcal{R}.
  \end{equation*}
  Then $R$ is an ICER (because intersection of ICERs are ICERs).
  It remains to show that $X/R$ is equicontinuous.
  The product system $\prod_{R^\prime \in \mathcal{R}} X/R^\prime$ is equicontinuous, since products of equicontinuous systems are equicontinuous.
  The map $(X/R, T) \ni  [x]_R  \mapsto ([x]_{R^\prime})_{R^\prime \in \mathcal{R}}$ is an isomorphism of systems and uniformly continuous and so $(X/R,T)$ is isomorphic to a subsystem of an equicontinous system and therefore equicontinuous.
  $X/R$
\end{proof}
