\section{Maximal Equicontinuous Factor}
\subsection{Definition, Existence, Uniqueness, Universal Property}
\begin{lemma}
  \label{lem:countableSubIntersection}
  Let $(X,d)$ be a compact metric space and $(A_j)_{j \in J}$ a family of closed sets in $X$.
  Then there exists a countable subset $I \subset J$ with
  \begin{equation*}
    \bigcap_{j \in J} A_j = \bigcap_{i \in I} A_{i}.
  \end{equation*}
\end{lemma}
\begin{proof}

  It is clear that $\bigcap_{j \in J} A_j \subset \bigcap_{i \in I} A_{i}$.
  It is left to show $\bigcap_{j \in J} A_j \supset \bigcap_{i \in I} A_{i}$:
Let $A:=\bigcap_{j \in J} A_j$.
  For $n \in \mathbb{N}$ let $U_n := \bigcup_{ x \in A} B(x, 1/n)$ and $C_n := X \setminus U_n$.
  Let $n \in \mathbb{N}$. Then
  \begin{equation*}
    \varnothing =  A \cap C_n.
  \end{equation*}
  Note that $C_n$ is compact, because it is a closed subset of a compact space.
  $(X \setminus A_j)_{j \in J}$ is an open cover of $C_n$,
  because
  \begin{equation*}
    C_n \subset X \setminus A = X \setminus \bigcap_{j \in J} A_j = \bigcup_{j \in J} X \setminus A_j
  \end{equation*}
  and so there exists a finite set $J_n \subset J$ with
  \begin{equation*}
    X \setminus U_n = C_n \subset \bigcup_{j \in J_n} X \setminus A_j = X \setminus \bigcap_{j \in J_n} A_j
  \end{equation*}
  and so
  \begin{equation*}
 \bigcap_{j \in J_n} A_j \subset U_n.
  \end{equation*}
  Now set $I := \bigcup_{n \in \mathbb{N}} I_n$.
  Then $I$ is countable being a countable union of finite sets.
  Furthermore
  \begin{equation*}
    \bigcap_{i \in I} A_i \subset \bigcap_{n \in \mathbb{N}} U_n = A.
  \end{equation*}
  For the last equality: It is clear that $\bigcap_{n \in \mathbb{N}} U_n \supset A$.
  To show $\bigcap_{n \in \mathbb{N}}U_n \subset A$:
  Assume (towards a contradiction) that $x \in \bigcap_{n \in \mathbb{N}}U_n $ with $x \notin A$.
  Since $A$ is closed there exists $m \in \mathbb{N}$ with $B(x,1/m) \subset X \setminus A$.
  But then for every $y \in A$:
  \begin{equation*}
    d(y,x) \geq 1/m
  \end{equation*}
  and so $x \notin \bigcap_{n \in \mathbb{N}}U_n$ contradicting the assumption.
\end{proof}
\begin{definition}[Maximal equicontinuous factor]
  Let $(X,T)$ be a TDS.
  A factor $\pi : (X,T) \to (Y,T)$ is called a \emph{maximal equicontinuous factor} (MEF) of $(X,T)$ if and only if $\pi$ is equicontinuous and maximal in the following sense:
  Let $R$ be the ICER generated by $\pi$.
  If $\varphi : (X,T) \to (Y,T)$ is another equicontinuous factor and $R_\varphi$ the ICER generated by $\varphi$, then $R \subset R_\varphi$.
\end{definition}
\begin{theorem}[Existence and uniqueness of the MEF]
  Let $(X,T)$ be a TDS.
  Then $(X,T)$ has a maximal equicontinuous factor and any two maximal equicontinuous factors are equivalent.
  We therefore simply speak of the maximal equicontinuous factor.
  \end{theorem}
\begin{proof}
  Existence:
  Let
  \begin{equation*}
    \mathcal{R} := \{ R \subset X \times X : R \ \text{ICER with} \ X/R \ \text{equicontinuous}\}.
  \end{equation*}
  Then $ X \times X \in \mathcal{R}$ and so $\mathcal{R} \neq \varnothing$.
    Now let 
  \begin{equation*}
    R:= \cap \mathcal{R}.
  \end{equation*}
  Then $R$ is an ICER (because intersection of ICERs are ICERs).
  If $\varphi : (X,T)\to (Y,T)$ is a factor with $(Y,T)$ equicontinuous and $R_\varphi$ the ICER generated by $\varphi$. Then $(X/R_\varphi,T)$ is conjugate to $(Y,T)$ by \cref{prop:factorsAndIcers}.
  And so $(X/R_\varphi,T)$ is equicontinuous, because equicontinuity is conjugation invariant (\cref{prop:equiContConjugationInvariance}).
  Therefore $R \subset R_\varphi$ and so $R$ is the maximal ICER in the sense of the theorem.\\
  It is left to show, that $R$ is the ICER generated by an equicontinuous factor:
  By metrizability there exists a family $(R_n)_{n\in\mathbb{N}}$ of elements of $\mathcal{R}$ with
  $R = \cap_{n=1}^\infty R_n$ by \cref{lem:countableSubIntersection}.
  The product system $(\prod_{n=1}^\infty X/R_n,T)$ is equicontinuous, since products of equicontinuous systems are equicontinuous (\cref{prop:productsOfEquiContAreEquiCont}).
  Consider the morphism $f:= X \ni x  \mapsto ([x]_{R_n})_{n \in \mathbb{N}}$.
  Let $A:= f(X)$, then $A$ is $T$-invariant being the image of a morphism.
  Furthermore $A$ is compact by continuity of $f$ and compactness of $X$.
  Since $A$ is a compact subset of a Hausdorff space it follows that $A$ is closed.
  Therefore we can consider the subsystem $(A,T)$ which is an equicontinuous system because subsystems of equicontinuous systems are equicontinuous (\cref{prop:SubsystemOfEquiContIsEquiCont}).
  Consider the map $\pi: X \to  A , \  x  \mapsto ([x]_{R_n})_{n \in \mathbb{N}}$.
  Then $\pi$ is a factor and the ICER generated by $\pi$ is equal to $R$, because: $\forall x_1,x_2 \in X$:
  \begin{equation*}
    \pi(x_1) = \pi (x_2) \Leftrightarrow ([x_1]_{R_n})_{n\in \mathbb{N}} = ([x_2]_{R_n})_{n\in \mathbb{N}}
    \Leftrightarrow \forall n \in \mathbb{N} : (x_1, x_2 ) \in R_n
    \Leftrightarrow (x_1,x_2) \in \cap_{n=1}^\infty R_n = R.
  \end{equation*}
  Therefore $\pi$ is a maximal equicontinuous factor.\\
  Uniqueness:
  If $\varphi$ is another maximal equicontinuous factor and $R_\varphi$ the ICER generated by it, then $R \subset R_\varphi$ as was shown above.
  By maximality of $\varphi$: $R_\varphi \subset R$ and so $R_\varphi = R$.
\end{proof}
\begin{proposition}[MEF of equicontinuous TDS is trivial]
  Let $(X,T)$ be an equicontinuous TDS, then the MEF of $(X,T)$ is the identity $I: (X,T) \to (X,T)$.
\end{proposition}
\begin{proof}
  Evidently $I$ is an equicontinuous factor.
  Let $\Delta \subset X \times X$ be the diagonal. Then $\Delta$ is the ICER generated by $I$.
  Any ICER contains $\Delta$ (because equivalence relations are reflexive).
  Therefore $I$ is also maximal.
\end{proof}


\begin{proposition}[Universal Property of the MEF]
  Let $(X,T)$ be a TDS. Let $\pi : (X, T) \to  (X_m,T)$ be the MEF.
  Then for every equicontinuous factor $\varphi : (X,T) \to (Y,T)$ there exists a unique factor $\tilde{\varphi}: (X_m,T) \to (Y,T)$ with $\varphi = \tilde{\varphi} \circ \pi$.
  The proposition is summarized by the following diagram:
  \begin{center}
    \begin{tikzcd}[sep=large]
      (X,T) \arrow[d, "\pi"'] \arrow[r, "\varphi \ \text{equic.}"]  \arrow[dr, phantom, "\circlearrowleft", near start] 
      & (Y,T) \\
      (X_m,T)  \arrow[ur,"\exists ! \tilde{\varphi}"'] & \phantom{a} \\
    \end{tikzcd}
  \end{center}
  \end{proposition}
\begin{proof}
  Let $R$ be the ICER generated by $\pi$.
  Wlog $X_m = X/R$ and $\pi$ is the quotient map.
  Let $(x_1, x_2) \in R$. Then $\varphi (x_1) = \varphi (x_2)$, because $R$ is contained in the ICER generated by $\varphi$.
  The universal property of the quotient topology implies that there exists a unique continuous map $\tilde{\varphi}: X/R \to Y$ with $\varphi = \tilde{\varphi}\circ \pi$.
  It is left to show that $\tilde{\varphi}$ is a factor.
  Clearly $\tilde{\varphi}$ is surjective and for all $t\in T$ and $x \in X$:
  \begin{equation*}
    t \tilde{\varphi} (\pi(x)) =  t \varphi (x) = \varphi (tx) = \tilde{\varphi}(\pi (tx)) = \tilde{\varphi}(t \pi (x)) .
  \end{equation*}
  Showing that $\tilde{\varphi}$ is a homomorphism.
\end{proof}

\subsection{Computing the MEF}
In this section let $(X,T)$ be a TDS with $T$ abelian.
Let $T^*$ be the set of all (continuous) characters of $T$.
\begin{definition}[Koopman operator]
  The \emph{Koopman operator} $U : T \to L(C(X))$ of $(X,T)$ is defined by $(U(t) f)(x) := f(t^{-1}x)$.
  $0 \neq f \in C(X)$ is called an \emph{eigenfunction} (of $U$) to \emph{eigenvalue} $\chi \in T^*$ if and only if $f \in \ker (U-\chi):= \bigcap_{t \in T} \ker (U(t)- \chi (t)I)$.
\end{definition}
\begin{proposition}
  Assume $(X,T)$ minimal and $x_0 \in X$ arbitary. Let $F$ be the set of eigenfunctions normalized such that they have value $1$ at $x_0$ and $V$ the set of eigenvalues. Let $J: V \to F$, $J(\chi):=$ the unique continuous extension of the map $T x_0 \ni t x_0 \mapsto \chi(t^{-1})$. Then $J$ is a bijection.
\end{proposition}
\begin{proof}
  Injectivity is obvious.
  To show surjectivity:
  Let $f \in F$. Let $\chi \in V$ be the corresponding eigenvalue and $t \in T$. Then $ f(t x_0) = (U(t^{-1}) f)(x_0) = \chi(t^{-1}) f(x_0) = \chi (t^{-1})$. Therefore $J(\chi ) = f$.
\end{proof}
\begin{definition}[Discrete spectrum]
  \begin{equation*}
    (X,T) \ \text{has \emph{discrete spectrum}}:\Leftrightarrow C(X) = \clin \bigcup_{\chi \in T^*} \ker (U- \chi).
  \end{equation*}
\end{definition}
\begin{theorem}
  \label{mef:thm:equiEQdiscrete}
  $(X,T)$ is equicontinuous $\Leftrightarrow$ $(X,T)$ has discrete spectrum.
\end{theorem}
\begin{proof}
  See \cite{HK2023} Theorem 2.11.
\end{proof}

\begin{theorem}[MEF eigenfunction characterisation]
  \label{thm:MEF_EFchar}
  Let $x_1,x_2 \in X$. Let $\pi : (X,T) \to (X_m,T)$ be the MEF.
  Then
  \begin{equation*}
  \pi (x_1) = \pi (x_2) \Leftrightarrow 
    \forall f \in \bigcup_{\chi \in T^*} \ker (U- \chi) : f(x_1) = f(x_2).
  \end{equation*}
\end{theorem}
\begin{proof}
  To \enquote{$\Rightarrow$}:
  Let $f \in \bigcup_{\chi \in T^*} \ker (U- \chi)$ be an eigenfunction and $\chi$ the corresponding eigenvalue.
  $(f(X),T)$ is an equicontinuous factor, where the action of $t \in T$ on $f(X)$ is given by multiplication with $\chi(-t)$ and $f(X)$ carries the metric induced from $\mathbb{C}$ (the equicontinuity follows from $|\chi|=1$).
  For all $t \in T, x \in X$:
  \begin{equation*}
    t f(x) = \chi (-t) f (x) = (U(-t) f)(x) = f(tx).
  \end{equation*}
  Therefore $f: (X,T) \to (f(X),T)$ is an equicontinuous factor.
  By the universal property of the MEF there exists a continuous $\tilde{f}  : X_m  \to \mathbb{C}$ with $\tilde{f} \circ \pi = f$.
  Therefore $f(x_1) = \tilde{f} (\pi (x_1) )=  \tilde{f} ( \pi (x_2)) = f (x_2)$.\\
  To \enquote{$\Leftarrow$}:
  Assume $\pi (x_1) \neq \pi (x_2)$.
  Then there exists an eigenfunction $h \in C(X_m)$ of $(X_m,T)$ with $h (\pi(x_1)) \neq h(\pi (x_2))$.
  Assume otherwise: Then by theorem \ref{mef:thm:equiEQdiscrete} $g (\pi (x_1)) = g(\pi (x_2))$ for every $g \in C(X_m)$ contradicting the Hausdorffness of $X_m$.
  Now $f:= h \circ \pi$ is an eigenfunction of $(X,T)$ with $f(x_1) \neq f(x_2)$ contradicting the assumption.
\end{proof}

\begin{proposition}[MEF of skew rotation]
Let $a \in \mathbb{T}$ and $\alpha: \mathbb{T}^2 \to \mathbb{T}^2,$ $\alpha (x,y) := (ax,xy)$.
  Consider the TDS $(\mathbb{T}^2,\mathbb{Z})$ with the $\mathbb{Z}$-action defined by $t \cdot (x,y) := \alpha^t (x,y)$.
  Then the MEF is given by $\pr_1 : (\mathbb{T}^2,\mathbb{Z}) \to (\mathbb{T},\mathbb{Z}), \pr_1 (x,y) := x$, where the $\mathbb{Z}$-action on $\mathbb{T}$ is defined by $t\cdot  x := a^t x$.
\end{proposition}
\begin{proof}
  Well-definedness of the TDS:
  Clearly $\mathbb{T}^2$ is a compact metric space and $\mathbb{Z}$ a topological group.
  $\alpha$ is a homeomorphism with inverse $\alpha^{-1} (x,y):= (a^{-1} x, a x^{-1} y)$.
  This shows the well-definedness of the $T$-action.
  The continuity of the $T$-action follows from continuity of $\alpha$ and $\alpha^{-1}$ together with the fact that $\mathbb{Z}$ carries the discrete topology.
  We want to apply \cref{thm:MEF_EFchar}.
  Let $\pi : (\mathbb{T}^2 , \mathbb{Z}) \to (X_m, \mathbb{Z})$ be the MEF.
  Let $(x_1, y_1), (x_2, y_2) \in \mathbb{T}^2$.
  If $x_1 \neq x_2$, then $\pr_1$ is an eigenfunction to eigenvalue $t \mapsto a^{-t}$ that takes different values on $x_1$ and $x_2$ and so $\pi(x_1) \neq \pi(x_2)$.
  Assume $x_1 = x_2$.
  Let $f \in C(\mathbb{T}^2)$ be an eigenfunction with eigenvalue $\chi$.
  The goal is to show $f(x_1, y_1) = f(x_2,y_2)$.
  Let $\lambda := \chi (-1)$.
  We consider $f$ as an element of $L^2( \mathbb{T}^2)$.
  Let $(e_n)_{n \in \mathbb{Z}}$, \ $e_n(x):= x^n$ be the monomials on $\mathbb{T}$.
  They form an orthonormal basis of $L^2(\mathbb{T})$ and so $(e_n \otimes e_m)_{(n,m) \in \mathbb{Z}^2}$ is an orthonormal basis of $L^2(\mathbb{T}^2)$.
  Therefore (in $L^2$)
  \begin{equation*}
    f = \sum_{(n,m) \in \mathbb{Z}^2} \langle f, e_n \otimes e_m \rangle e_n \otimes e_m.
  \end{equation*}
  Furthermore $\alpha$ and $\alpha^{-1}$ preserve the Haar measure of $\mathbb{T}^2$ therefore we can lift the eigenvalue equation to $L^2$:
  \begin{equation}
    \label{eq:T2mefEveq}
    \lambda f = \chi(-1) f=  U(-1)f  = f \circ \alpha.
  \end{equation}
  Now for $(n,m) \in \mathbb{Z}^2$ and $(x,y) \in \mathbb{T}^2$:
  \begin{equation*}
    e_n \otimes e_m (\alpha (x,y)) = e_n \otimes e_m (a x, x y) = (ax)^n (xy)^m = a^n x^{m+n} y^m
    = a^n e_{m+n} \otimes e_m (x,y).
  \end{equation*}
  Therefore \cref{eq:T2mefEveq} is equivalent to $\forall (n,m) \in \mathbb{Z}^2$:
  \begin{equation*}
    \sum_{(n,m) \in \mathbb{Z}^2} \lambda \langle f, e_n \otimes e_m \rangle e_n \otimes e_m = 
    \sum_{(n,m) \in \mathbb{Z}^2} a^n \langle f, e_n \otimes e_m \rangle e_{n+m} \otimes e_m.
  \end{equation*}
  Which implies that
$\forall (n,m) \in \mathbb{Z}^2$:
\begin{equation*}
  |\langle f, e_{n+m} \otimes e_m \rangle|  = |\langle f, e_n \otimes e_m \rangle|.
\end{equation*}
  Fix $m \in \mathbb{Z} \setminus \{0\}$.
  We want to show that $\forall n \in \mathbb{Z}: |\langle f, e_n \otimes e_m \rangle| = 0$.
  Let $n \in \mathbb{Z}$ with $|\langle f, e_n \otimes e_m \rangle| > 0$.
  Then for all $k \in \mathbb{N}: |\langle f, e_{n+km} \otimes e_m \rangle| = |\langle f, e_n \otimes e_m \rangle|>0$.
  Then
  \begin{equation*}
    \infty = \sum_{k \in \mathbb{N} } |\langle f, e_{n+km} \otimes e_m \rangle|^2
    \leq   \sum_{(n,m) \in \mathbb{Z}^2} |\langle f, e_{n} \otimes e_m \rangle|^2 
    =\|f\|_{L^2} < \infty.
  \end{equation*}
  A contradiction.
  In total
  \begin{equation*}
    f = \sum_{n \in \mathbb{Z}} \langle f, e_n \otimes e_0 \rangle e_n \otimes e_0.
  \end{equation*}
  Therefore there is $h \in L^2(\mathbb{T})$ and a nullset $N \subset \mathbb{T}^2$ so that for all $(x,y) \in \mathbb{T}^2 \setminus N : f(x,y) = h(x)$.
  Therefore for all  $(x,y) , (x,y^\prime) \in \mathbb{T}^2 \setminus N , f(x,y) = f(x,y^\prime)$. 
  By continuity of $f$ (and a small null-set argument) this implies that $\forall (x,y), (x,y^\prime) \in \mathbb{T}^2 : f(x,y) = f(x,y^\prime)$.
  In particular this shows, that $f(x_1,y_1) = f(x_2, y_2)$ since $x_1 = x_2$.
\end{proof}

\subsection{MEF vs. Kronecker subsystem}
Let $(X,T)$ be a TDS with $T$ abelian and $\mu$ a $T$-invariant Borel probability measure on $X$.
Then $(X,T,\mu)$ is a concrete measure preserving system (MPS).
We have the MPS Koopman operator
\begin{equation*}
  \begin{split}
    &V : T \to L(L^2(X)),\\
    & (V(t) f)(x) := f (t^{-1} x). 
  \end{split}
\end{equation*}
According to the ISEM lectures
  $(X,T,\mu)$ has discrete spectrum if and only if
  \begin{equation*}
    L^2(X) = \clin^{L^2} \bigcup_{\chi \in T^*} \ker V - \chi.
  \end{equation*}
  \begin{proposition}
    Let $(X,T)$ be a TDS with $T$ abelian and $\mu$ a $T$-invariant Borel probability measure on $X$.
  If $(X,T)$ has TDS discrete spectrum, then $(X,T,\mu)$ has MPS discrete spectrum.
  \end{proposition}
  \begin{proof}
    Consider the following:
  \begin{itemize}
    \item continuous eigenfunctions are also $L^2$-eigenfunctions,
    \item $C(X)$ is dense in $L^2(X)$,
    \item $C(X)$ convergence implies convergence in $L^2(X)$, because $\mu$ is a finite measure.
  \end{itemize}
  \end{proof}
  Let $(X,T)$ be a TDS with $T$ abelian and $\mu$ a $T$-invariant Borel probability measure on $X$.
    Let $\pi : (X,T) \to (X_m,T)$ be the MEF.
  Then $\nu:= \pi_* \mu$ is a $T$-invariant probability measure on $X_m$ and therefore $(X_m,T,\nu)$ is also a concrete MPS, which has discrete spectrum (in the MPS sense) and $\pi$ is a factor (of concrete MPS).
  Let $\mathcal{X}$/$\mathcal{X}_m$ be the measure preserving system associated to $(X,T,\mu)$/$(X_m,T,\nu)$ and $\Pi: \mathcal{X}_m \to \mathcal{X}$ the associated extension.
  Let $K: \mathcal{K} \to \mathcal{X}$ be the Kronecker subsystem of $\mathcal{X}$.
  Since $\mathcal{X}_m$ has discrete spectrum we can apply the universal property of the Kronecker subsystem to obtain an extension $J: \mathcal{X}_m \to \mathcal{K}$ such that the following diagram commutes:
  \begin{center}
    \begin{tikzcd}[sep=large]
      \mathcal{X}_m \arrow[d, "J"'] \arrow[r, "\Pi"]  \arrow[dr, phantom, "\circlearrowleft", near start] 
      & \mathcal{X}  \\
      \mathcal{K}  \arrow[ur,"K"'] & \phantom{a} \\
    \end{tikzcd}
  \end{center}

