\section{Preliminaries}
\begin{definition}[Topological dynamical system]
  A \emph{topological dynamical system} (TDS) is a triplet $(X,T,\cdot)$, where
  \begin{itemize}
    \item $X$ is a topological Hausdorff space,
    \item $T$ is a topological group,
    \item $\cdot : X \times T \to X$ is a continuous group action, that is $\forall x \in X, \ s,t \in T$: 
    \begin{enumerate}
      \item $ x\cdot e = x$,
      \item $(x\cdot s)\cdot t = x \cdot (s \cdot t)$.
    \end{enumerate}
  \end{itemize}
\end{definition}
\begin{remark}
  The map $\cdot$ is usually supressed in the notation.
\end{remark}

\begin{definition}[Morphism of TDS]
  Let $(X,T), \ (Y,T)$ be topological dynamical systems.
  A map $\pi : X \to Y$ is called a \emph{homomorphism} from $(X,T) \to (Y,T)$ if $\pi$ is continuous and $\forall x \in X, t \in T$:
  \begin{equation*}
     \pi (x)   t = \pi ( x t).
  \end{equation*}
\end{definition}
\begin{definition}[Factor of TDS]
   Let $(X,T), \ (Y,T)$ be topological dynamical systems and $\pi: X \to Y$ a homomorphism.
   Then $\pi$ is called a \emph{factor map} if it is surjective.
   If there exists a factor map $X \to Y$ then $Y$ is called a \emph{factor} of $X$.
\end{definition}

\begin{proposition}[Quotient system]
  Let $(X,T)$ be a TDS with $X$ compact. Let $R \subset X \times X$ be a $T$-invariant\footnote{that is $\forall t \in T, \forall (x,x^\prime ) \in R : (xt ,x^\prime t) \in R$} and closed equivalence relation (ICER for short).
  Let $\pi : X \to X/R$ be the quotient map.
  Endow $X/R$ with the quotient topology, that is $U \subset X/R$ open $: \Leftrightarrow \pi^{-1}(U)$ is open.
  Define the $T$-action by $[x] t := [x t]$.
  Then $(X/R, T)$ is a topological dynamical system with $X/R$ compact and $\pi$ is a factor map.
\end{proposition}
\begin{proof}
 The quotient $R/F$ is a Hausdorff space, because $X$ is a compact Hausdorff space and $R$ is closed.
 The compactness of $R/F$ follows from $R/F = \pi (X)$ and the compactness of $X$ together with the continuity of $\pi$. The $T$-invariance of $R$ implies the well definedness of the $T$-action.
\end{proof}

\begin{definition}[Product system]
  Let $(X_\alpha,T, \cdot_\alpha)_{\alpha \in A}$ be a family of topological dynamical systems.
  Then the \emph{product system} $(X, T, \cdot)$ of the family is defined by
  \begin{equation*}
    X := \prod_{\alpha \in A} X_\alpha, \quad (x\cdot t)_{\alpha} := x \cdot_\alpha t.
  \end{equation*}
\end{definition}
