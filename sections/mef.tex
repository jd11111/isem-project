\section{Maximal Equicontinuous Factor}

\begin{theorem}[Maximal equicontinuous factor]
  Let $(X,T)$ be a TDS.
  Then $(X,T)$ has a unique (up to equivalence) equicontinuous factor $\pi : (X,T) \to (A,T)$ that is maximal in the following sense:
  Let $R$ be the ICER generated by $\pi$.
  If $\varphi : (X,T) \to (Y,T)$ is another equicontinuous factor and $R_\varphi$ the ICER generated by $\varphi$, then $R \subset R_\varphi$.
  This factor is called the \emph{maximal equicontinuous factor} (MEF).
\end{theorem}
\begin{proof}
  Existence:
  Let
  \begin{equation*}
    \mathcal{R} := \{ R \subset X \times X : R \ \text{ICER with} \ X/R \ \text{equicontinuous}\}.
  \end{equation*}
  Then $ X \times X \in \mathcal{R}$ and so $\mathcal{R} \neq \varnothing$.
    Now let 
  \begin{equation*}
    R:= \cap \mathcal{R}.
  \end{equation*}
  Then $R$ is an ICER (because intersection of ICERs are ICERs).
  If $\varphi : (X,T)\to (Y,T)$ is a factor with $(Y,T)$ equicontinuous and $R_\varphi$ the ICER generated by $\varphi$, then $X/R_\varphi$ is also equicontinuous, since $(X/R_\varphi,T)$ is conjugate to $(Y,T)$ {\color{red} cite prop}.
  Therefore $R \subset R_\varphi$ and so $R$ is the maximal ICER in the sense of the theorem.\\
  It is left to show, that $R$ is the ICER generated by an equicontinuous factor:
  By metrizability there exists a family $(R_n)_{n\in\mathbb{N}}$ of elements of $\mathcal{R}$ with
  $R = \cap_{n=1}^\infty R_n$ {\color{red} do lemma}.
  The product system $(\prod_{n=1}^\infty X/R_n,T)$ is equicontinuous, since products of equicontinuous systems are equicontinuous {\color{red}(cite prop ...)}.
  Consider the morphism $f:= X \ni x  \mapsto ([x]_{R_n})_{n \in \mathbb{N}}$.
  Let $A:= f(X)$, then $A$ is $T$-invariant being the image of a morphism.
  Furthermore $A$ is compact by continuity of $f$ and compactness of $X$.
  Since $A$ is a compact subset of a Hausdorff space it follows that $A$ is closed.
  Therefore we can consider the subsystem $(A,T)$ which is an equicontinuous system because subsystems of equicontinuous systems are equicontinuous {\color{red}( cite prop ...)}.
  Consider the map $\pi: X \to  A , \  x  \mapsto ([x]_{R_n})_{n \in \mathbb{N}}$.
  Then $\pi$ is a factor and the ICER generated by $\pi$ is equal to $R$, because: $\forall x_1,x_2 \in X$:
  \begin{equation*}
    \pi(x_1) = \pi (x_2) \Leftrightarrow ([x_1]_{R_n})_{n\in \mathbb{N}} = ([x_2]_{R_n})_{n\in \mathbb{N}}
    \Leftrightarrow \forall n \in \mathbb{N} : (x_1, x_2 ) \in R_n
    \Leftrightarrow (x_1,x_2) \in \cap_{n=1}^\infty R_n = R.
  \end{equation*}
  Therefore $\pi$ is a maximal equicontinuous factor.\\
  Uniqueness:
  If $\varphi$ is another maximal equicontinuous factor and $R_\varphi$ the ICER generated by it, then $R \subset R_\varphi$ as was shown above.
  By maximality of $\varphi$: $R_\varphi \subset R$ and so $R_\varphi = R$.
\end{proof}

\begin{proposition}[Universal Property of the MEF]
  Let $(X,T)$ be a TDS. Let $\pi : (X, T) \to  (X/R,T)$ be the MEF.
  Then for every equicontinuous factor $\varphi : (X,T) \to (Y,T)$ there exists a unique factor $\tilde{\varphi}: (X/R,T) \to (Y,T)$ with $\varphi = \tilde{\varphi} \circ \pi$.
\end{proposition}
\begin{proof}
  Let $(x_1, x_2) \in R$. Then $\varphi (x_1) = \varphi (x_2)$, because $R$ is contained in the ICER generated by $\varphi$.
  The universal property of the quotient topology implies that there exists a unique continuous map $\tilde{\varphi}: X/R \to Y$ with $\varphi = \tilde{\varphi}\circ \pi$.
  It is left to show that $\tilde{\varphi}$ is a factor.
  Clearly $\tilde{\varphi}$ is surjective and for all $t\in T$ and $x \in X$:
  \begin{equation*}
    t \tilde{\varphi} (\pi(x)) =  t \varphi (x) = \varphi (tx) = \tilde{\varphi}(\pi (tx)) = \tilde{\varphi}(t \pi (x)) .
  \end{equation*}
  Showing that $\tilde{\varphi}$ is a homomorphism.
\end{proof}

\begin{example}
  $(\mathbb{Z}, \mathbb{T}^2)$, $a \in \mathbb{T}$ not a root of unity, $\alpha: \mathbb{T}^2 \to \mathbb{T}^2,$ $\alpha (x,y) := (ax,xy)$ and $t (x,y) := \alpha^t (x,y)$.
  Is minimal and distal (somewhere in Auslander).
  \begin{equation*}
    \widehat{\mathbb{Z}} =  \{  \lambda^\bullet  : \lambda \in \mathbb{T}\}
  \end{equation*}
  Continuous Eigenfunction: $f: \mathbb{T}^2 \to \mathbb{C}$. ...
  Eigenvalues powers of $a$?
\end{example}
