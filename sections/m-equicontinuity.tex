\section{$m$-Equicontinuity}

\begin{definition}[$m$-Equicontinuity]
	Fix $m \in \mathbb{N}$. Then $(X, T)$ is \emph{$m$-equicontinuous} if for all $\varepsilon > 0$ there exists a $\delta > 0$, such that for all $U \subseteq X$ open with $\operatorname{diam}(U) < \delta$, $x_1, \dots, x_m \in U$ and $t \in T$ there are $i, j \in \{1, \dots, m\}$ with $i \neq j$ such that $d(tx, ty) < \varepsilon$.
\end{definition}

\begin{proposition}
	If $(X, T)$ is $m$-equicontinuous for some $m \in \mathbb{N}$ it is also $m'$-equicontinuous for all $m' > m$.
\end{proposition}
\begin{proof}
	This is clear from the definition.
\end{proof}

\begin{definition}[$m$-Sensitivity]
	Fix $m \in \mathbb{N}$. Then $(X, T)$ is \emph{$m$-sensitive} if there exists a $\varepsilon > 0$ such that for all non-empty open $U \subseteq X$ there exist $x_1, \dots, x_m \in U$ and $t \in T$ such that $d(tx_i, tx_j) \geq \varepsilon$ for all $i, j \in \{1, \dots, m\}$ with $i \neq j$.
\end{definition}

\begin{theorem}
	Fix $m \in \mathbb{N}$ and assume that $(X, T)$ is minimal. Then $(X, T)$ is either $m$-equicontinuous or $m$-sensitive.
\end{theorem}


\begin{example}
	Let $o$ be the dyadic odometer on $X = \{0, 1\}^\mathbb{N}$ defined in the first section. For $p \in \mathbb{N}$ define $a: X \times \mathbb{Z}/p\mathbb{Z} \to X \times \mathbb{Z}/p\mathbb{Z}$ by
	\begin{align*}
		a(x_1x_2\dots, z + p\mathbb{Z}) := (o(x_1x_2\dots), z + \min \{n \in \mathbb{N};\ a_n = 0\} + p\mathbb{Z}
	\end{align*}
	for all $(x_1x_2\dots, z + p\mathbb{Z}) \in X \times \mathbb{Z}/p\mathbb{Z}$. This map gives rise to a minimal TDS $(X \times \mathbb{Z}/p\mathbb{Z}, \mathbb{Z})$ which is $m$-equicontinuous if and only if $p < m$.
\end{example}
\begin{proof}
	todo
\end{proof}

\begin{definition}[$m$-MEF]
	Fix $m \in \mathbb{N}$ with $m \geq 2$. Assume that $(X, T)$ is minimal and let its MEF be given by $\pi_{eq}: X \to X_{eq}$. A factor $\pi: X \to Y$ is called an \emph{$m$-MEF} if for all $A \in X/R_{pi_{eq}}$ there are $B_1, \dots, B_{m-1} \in X/R_\pi$ such that
	\begin{align*}
		A = \bigcup_{i=1}^{m-1} B_i.
	\end{align*}
\end{definition}

\begin{remark}
	Note that the $B_i$ in the definition of an $m$-MEF need not be pairwise distinct, in particular any $m$-MEF is also an $m'$-MEF for all $m' > m$.
\end{remark}

\begin{definition}[$m$-Regionally Proximal Realtion]
	Let $(X, T)$ be a TDS and $m \geq 2$.
	A tupel $(x_1, \dots, x_m) \in X^m$ is called $m$-regionally proximal if for each $\varepsilon > 0$
	there are $(x_1', \dots, x_m') \in B_\varepsilon((x_1, \dots, x_m))$ and $t \in T$ such that $d(tx_i', tx_j') < \varepsilon$ for all $i, j \in \{1, \dots, m\}$.
	We denote the set of $m$-regionally proximal tupels by $Q_m$.
\end{definition}

\begin{remark}
	In the case of $m = 2$ the definition of the $m$-regionally proximal relation and the regionally proximal relation coincide,
	so the notation $Q_m$ is justified.
\end{remark}

\begin{proposition}
	Fix $m \in \mathbb{N}$ with $m \geq 2$. Let $(Y, T)$ be an $m$-MEF of $(X, T)$. Then $(Y, T)$ is $m$-equicontinuous
\end{proposition}

\begin{proposition}
	Assume that $(X, T)$ is minimal and that its MEF is finite. Then the MEF is the only $m$-MEF of $(X, T)$ for all $m \in \mathbb{N}$ with $m \geq 2$.
\end{proposition}
\begin{proof}
	Denote by $R_{eq}$ the ICER corresponding to the MEF. Let $R$ be an $m$-MEF for $m \in \mathbb{N}$ with $m \geq 2$. Then for each $A \in X/R_{eq}$ there are $B_1, \dots B_{m-1} \in X/R$ such that $A = \bigcup_{i=1}^{m-1} B_i$, so $X/R$ is finite, hence it is equicontinuous. Therefore $R = R_{eq}$.
\end{proof}

\begin{theorem}
	Assume $(X, T)$ is minimal. Let a TDS $(Y, T)$ be $m$-equicontinuous for some $m \in \mathbb{N}$ with $m > 2$, and let $\pi: X \to Y$ be a factor map. Then there exists an $m$-MEF $\rho: X \to Z$ of $(X, T)$ and a unique factor map $\pi': X/R \to Y$ with $\pi = \pi' \circ \rho$.
\end{theorem}

\begin{example}
	Fix $p \in \mathbb{N}$ and look at $(X \times \mathbb{Z}/p\mathbb{Z}, \mathbb{Z})$ from above. For $m \in \mathbb{N}$ with $m \geq 2$ all $m$-MEFs (up to isomorphism) of $(X \times \mathbb{Z}/p\mathbb{Z}, \mathbb{Z})$ are given by $(X \times \mathbb{Z}/q\mathbb{Z}, \mathbb{Z})$ for $q \in \mathbb{N}$ with $q < m$ and $q$ divides $p$. The corresponding factor map $\pi_{pq}: X \times \mathbb{Z}/p\mathbb{Z} \to X \times \mathbb{Z}/q\mathbb{Z}$ is given by
	\begin{align*}
		\pi_{pq}(x, z + p\mathbb{Z}) := (x, z + q\mathbb{Z})
	\end{align*}
	for all $(x, z + p\mathbb{Z}) \in X \times \mathbb{Z}/p\mathbb{Z}$. In particular the MEF is given by
	\begin{align*}
		(X \times \mathbb{Z}/1\mathbb{Z}, \mathbb{Z}) \cong (X, \mathbb{Z}).
	\end{align*}
\end{example}
\begin{proof}
	todo
\end{proof}
