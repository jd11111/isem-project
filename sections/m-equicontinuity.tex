\section{$m$-Equicontinuity}

\begin{definition}[$m$-Equicontinuity]
	Fix $m \in \mathbb{N}$. Then $(X, T)$ is \emph{$m$-equicontinuous} if for all $\varepsilon > 0$ there exists a $\delta > 0$, such that for all $U \subseteq X$ open with $\operatorname{diam}(U) < \delta$, $x_1, \dots, x_m \in U$ and $t \in T$ there are $i, j \in \{1, \dots, m\}$ with $i \neq j$ such that $d(tx, ty) < \varepsilon$.
\end{definition}

\begin{proposition}
	If $(X, T)$ is $m$-equicontinuous for some $m \in \mathbb{N}$ it is also $m'$-equicontinuous for all $m' > m$.
\end{proposition}
\begin{proof}
	This is clear from the definition.
\end{proof}

\begin{definition}[$m$-Sensitivity]
	Fix $m \in \mathbb{N}$. Then $(X, T)$ is \emph{$m$-sensitive} if there exists a $\varepsilon > 0$ such that for all non-empty open $U \subseteq X$ there exist $x_1, \dots, x_m \in U$ and $t \in T$ such that $d(tx_i, tx_j) \geq \varepsilon$ for all $i, j \in \{1, \dots, m\}$ with $i \neq j$.
\end{definition}

\begin{theorem}
	Fix $m \in \mathbb{N}$ and assume that $(X, T)$ is minimal. Then $(X, T)$ is either $m$-equicontinuous or $m$-sensitive.
\end{theorem}

\begin{definition}[$m$-Regionally Proximal Relation]
	Let $(X, T)$ be a TDS and $m \geq 2$.
	A tupel $(x_1, \dots, x_m) \in X^m$ is called $m$-regionally proximal if for each $\varepsilon > 0$
	there are $(x_1', \dots, x_m') \in B_\varepsilon((x_1, \dots, x_m))$ and $t \in T$ such that $d(tx_i', tx_j') < \varepsilon$ for all $i, j \in \{1, \dots, m\}$.
	We denote the set of $m$-regionally proximal tupels by $Q_m$.
\end{definition}

\begin{remark}
	In the case of $m = 2$ the definition of the $m$-regionally proximal relation and the regionally proximal relation coincide,
	so the notation $Q_m$ is justified.
\end{remark}

\begin{definition}
Let $m \in \mathbb{N}$ and $X$ any set.
	We define
	\begin{equation*}
	\Delta^m(X) := \{ (x_1, \dots, x_m) \in X^m \ | \ \text{there are} \ i, j \in \{1, \dots, m\} \ \text{such that} \ x_i = x_j \}.
	\end{equation*}
\end{definition}

\begin{theorem}[Theorem 4.6 in \cite{Garcia-Ramos2024}]
	Let $(X, T)$ be a minimal TDS and $T$ abelian.
	Then $(X, T)$ is $m$-equicontinuous if and only if $Q_m(X) \setminus \Delta^m(X) = \varnothing$ for $2 \leq m \in \mathbb{N}$.
\end{theorem}
\begin{proof}
	Let $(x_1, \dots, x_m) \in Q_m(X) \setminus \Delta^m(X)$.
	We will show that $(X, T)$ is not equicontinuous.
	Let $A_1, \dots, A_m$ be disjoined closed neighborhoods of $x_1, \dots, x_m$ respectively.
	Define $\varepsilon = \min_{i, j} d(A_i, A_j)$.
	Let $\delta > 0$.
	Since $X$ is compact and minimal (so every orbit is dense) we find a finite cover $\bigcup_{g \in F} gB_\delta(x) = \bigcup_{g \in G} gB_\delta(x)$ of $X$ for some finite $F \subset G$ and $x \in X$ (the choice of $x$ does not matter).
	We define $\lambda$ to be the Lebesgue covering number of that finite cover.
	Because $(x_1, \dots, x_m) \in Q_m$ for all $i \neq j \in \{1, \dots, m\}$ we have that there is a $g_0 \in G$ and $x_j' \in A_i$ such that $g_0x_i'$ and $g_0 x_j'$ get arbitrarily close, in particular $d(g_0 x_i', g_0 x_j') < \lambda$.
	Therefore, the set $\{ g_0x_1', \dots, g_0x_m' \} \subset hB_\delta(x)$ for some $h \in G$ and $\{ h^{-1}g_0x_1', \dots, h^{-1}g_0x_m' \} \subset B_\delta(x)$.
	But now
	\[ d(g_0h(h^{-1g_0}x_i'), \dots, g_0h(h^{-1g_0}x_j')) = d(x_i', x_j') > d(A_1, A_m) \geq \varepsilon \]
	for all $i \neq j \in \{1, \dots, m\}$ contradicting $m$-equicontinuity since $\delta$ was arbitrary.

	For the converse statement let $(X, T)$ be a TDS that is not $m$-equicontinuous.
	Therefore, there exists an $\varepsilon > 0$ such that for all $n \in \mathbb{N}$ there are $(x_1, \dots, x_n) \in O_{n}$,
	where $\text{diam}(O_n) < \frac{1}{n}$, and $g^{(n)} \in G$ such that
	we have a sequence $(x_1^{(n)}, \dots, x_m^{(n)})_{n \in \mathbb{N}}$ with
	\[ d(g^{(n)}x_i^{(n)}, g^{(n)}x_j^{(n)}) > \varepsilon \]
 	for all $i \neq j \in \{1, \dots, m\}$ and $n \in \mathbb{N}$.
	Since $X$ is compact we have a converging subsequence and we can say that without loss of generality $g^{(n)}x_i^{(n)} \to y_i$ for all $i \in \{1, \dots, m\}$.
	Note that $(y_1, \dots, y_m)$ is not in $\Delta^m(X)$, since their distance is pairwise bounded from below by $\varepsilon$.
	We will show that $(y_1, \dots, y_m) \in Q_m(X)$.
	Let $\varepsilon' > 0$ and $N \in \mathbb{N}$ such that $d(g^{(N)}x_i^{N}, y_i) < \varepsilon'$ for all $i \in \{1, \dots, m\}$.
	Define $y_i' := g_k^{(N)} x_i^{(N)}$.
	Then we have
	\[ d((g^{(N)})^{-1} y_i', (g^{(N)})^{-1} y_j') = d(x_i^{(N)}, x_j^{(N)}) < \varepsilon' \]
	for all $i \neq j \in \{1, \dots, m\}$.
	And therefore $(y_1, \dots, y_m) \in Q_m(X)$.

	{ \color{red} where do we need $T$ to be abelian?}
 \end{proof}

\begin{example}
	Let $o$ be the dyadic odometer on $X = \{0, 1\}^\mathbb{N}$ defined in the first section. For $p \in \mathbb{N}$ define $a: X \times \mathbb{Z}/p\mathbb{Z} \to X \times \mathbb{Z}/p\mathbb{Z}$ by
	\begin{align*}
		a(x_1x_2\dots, z + p\mathbb{Z}) := (o(x_1x_2\dots), z + \min \{n \in \mathbb{N};\ x_n = 0\} + p\mathbb{Z})
	\end{align*}
	for all $(x_1x_2\dots, z + p\mathbb{Z}) \in X \times \mathbb{Z}/p\mathbb{Z}$. This map gives rise to a minimal TDS $(X \times \mathbb{Z}/p\mathbb{Z}, \mathbb{Z})$ which is $m$-equicontinuous if and only if $p < m$. \color{red} This example is wrong. The function $a$ is not defined everywhere and can not be extended continuously.
\end{example}
\begin{proof}
	We equip $X$ with the metric
	\begin{align*}
		d_X(x_1x_2\dots, y_1y_2\dots) := \inf\{\frac{1}{n};\ x_n \neq y_n\}
	\end{align*}
	for all $(x_1x_2\dots), (y_1y_2\dots) \in X$ and $\mathbb{Z}/p\mathbb{Z}$ with the discrete metric $d_d$. Then we equip $X \times \mathbb{Z}/p\mathbb{Z}$ with the metric
	\begin{align*}
		d((x, \alpha), (y, \beta)) := d_X(x, y) + d_d(\alpha, \beta)
	\end{align*}
	for all $(x, \alpha), (y, \beta) \in X \times \mathbb{Z}/p\mathbb{Z}$.
	
	Fix $m \in \mathbb{N}$ with $m \geq 2$. Let $\pi_X: X \times \mathbb{Z}/p\mathbb{Z} \to X$ be the projections onto the first component, which is clearly a factor. We prove that
	\begin{align*}
		Q_m = \{(u_1, \dots, u_m) \in (X \times \mathbb{Z}/p\mathbb{Z})^m;\ \pi_X(u_i) = \pi_X(u_j)\ \forall i, j \in \{1, \dots, m\}\}
	\end{align*}
	holds.
	
	First take $(u_1, \dots u_m) \in (X \times \mathbb{Z}/p\mathbb{Z})^m$ such that there are $i, j \in \{1, \dots, m\}$ such that $\pi_X(u_i) \neq \pi_X(u_j)$. Choose $\varepsilon < \frac{1}{3} d_X(\pi_x(u_i), \pi_X(u_j))$. Then for any $(u_1', \dots, u_m') \in B_\varepsilon(u_1, \dots, u_m)$ and any $n \in \mathbb{Z}$ we have
	\begin{align*}
		d(a^nu_i', a^nu_j') &\geq d_X(\pi_X(a^nu_i'), \pi_X(a^nu_j'))\\
		&= d_X(o^n\pi_X(u_i'), o^n\pi_X(u_j'))\\
		&= d_X(\pi_X(u_i'), \pi_X(u_j'))\\
		&\geq d(\pi_X(u_i), \pi_X(u_j)) - d(\pi_X(u_i), \pi_X(u_i')) - d(\pi_X(u_j'), \pi_X(u_j))\\
		&> 3\varepsilon - \varepsilon - \varepsilon = \varepsilon
	\end{align*}
	so $(u_1, \dots, u_m) \notin Q_m$.
	
	Now fix $(u_1, \dots, u_m) \in (X \times \mathbb{Z}/p\mathbb{Z})^m$ with $x = \pi_X(u_i)$ for all $i \in \{1, \dots, m\}$ and some $x \in X$ and let $\varepsilon > 0$ be given. There is some $k \in \mathbb{N}$ such that $\frac{1}{k} < \frac{\varepsilon}{m}$ and some $n \in \mathbb{N}$ such that $o^m x$ starts with $n$ zeroes.
\end{proof}

\begin{definition}[$m$-MEF]
	Fix $m \in \mathbb{N}$ with $m \geq 2$. Assume that $(X, T)$ is a minimal TDS and let its MEF be given by $\pi_{\mathrm{eq}}: X \to X_{\mathrm{eq}}$. Let $R_{\pi_{\mathrm{eq}}}$ be the ICER generated by $\pi_{eq}$. Let $\pi: (X,T) \to (Y,T)$ be a factor and $R_\pi$ the ICER generated by $\pi$.
	Then $\pi$ is called a \emph{$m$-MEF} if
	\begin{equation*}
		\forall A \in X/R_{\pi_{\mathrm{eq}}} \exists B_1, \dots, B_{m-1} \in X/R_\pi: 
		A = \bigcup_{i=1}^{m-1} B_i.
	\end{equation*}
\end{definition}

\begin{remark}
	Note that the $B_i$ in the definition of an $m$-MEF need not be pairwise distinct, in particular any $m$-MEF is also an $m'$-MEF for all $m' > m$.
\end{remark}

\begin{proposition}[Lemma 4.14 in \cite{Garcia-Ramos2024}]
	Fix $m \in \mathbb{N}$ with $m \geq 2$. Let $\pi : (X,T) \to (Y, T)$ be an $m$-MEF. Then $(Y, T)$ is $m$-equicontinuous.
\end{proposition}

\begin{proposition}
	Assume that $(X, T)$ is minimal and that its MEF is finite. Then the MEF is the only $m$-MEF of $(X, T)$ for all $m \in \mathbb{N}$ with $m \geq 2$.
\end{proposition}
\begin{proof}
	Denote by $R_{eq}$ the ICER corresponding to the MEF. Let $R$ be an $m$-MEF for $m \in \mathbb{N}$ with $m \geq 2$. Then for each $A \in X/R_{eq}$ there are $B_1, \dots B_{m-1} \in X/R$ such that $A = \bigcup_{i=1}^{m-1} B_i$, so $X/R$ is finite, hence it is equicontinuous. Therefore $R = R_{eq}$.
\end{proof}

\begin{theorem}
	Assume $(X, T)$ is minimal. Let a TDS $(Y, T)$ be $m$-equicontinuous for some $m \in \mathbb{N}$ with $m > 2$, and let $\pi: X \to Y$ be a factor map. Then there exists an $m$-MEF $\rho: X \to Z$ of $(X, T)$ and a unique factor map $\pi': X/R_\rho \to Y$ with $\pi = \pi' \circ \rho$.
\end{theorem}

\begin{example}
	Fix $p \in \mathbb{N}$ and look at $(X \times \mathbb{Z}/p\mathbb{Z}, \mathbb{Z})$ from above. For $m \in \mathbb{N}$ with $m \geq 2$ all $m$-MEFs (up to isomorphism) of $(X \times \mathbb{Z}/p\mathbb{Z}, \mathbb{Z})$ are given by $(X \times \mathbb{Z}/q\mathbb{Z}, \mathbb{Z})$ for $q \in \mathbb{N}$ with $q < m$ and $q$ divides $p$. The corresponding factor map $\pi_{pq}: X \times \mathbb{Z}/p\mathbb{Z} \to X \times \mathbb{Z}/q\mathbb{Z}$ is given by
	\begin{align*}
		\pi_{pq}(x, z + p\mathbb{Z}) := (x, z + q\mathbb{Z})
	\end{align*}
	for all $(x, z + p\mathbb{Z}) \in X \times \mathbb{Z}/p\mathbb{Z}$. In particular the MEF is given by
	\begin{align*}
		(X \times \mathbb{Z}/1\mathbb{Z}, \mathbb{Z}) \cong (X, \mathbb{Z}).
	\end{align*}
\end{example}
\begin{proof}
	todo
\end{proof}
