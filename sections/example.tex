\section{Example: Thue Morse Subshift}

We consider the finite alphabet $\Ac := \{ 0, 1 \}$.
The set of finite words is given as
\[ \Ac^* := \bigcup_{n\in\N} \Ac^n \]
and the infinite sequences are given by $\Ac^\N$.
Consider the following substitution
\[ T: \Ac \longrightarrow \Ac^*,\; a \longmapsto	\begin{cases} 01 & a=0 \\ 10 & a=1 \end{cases} \,.\]
This of course extends to a substitution
\[ \hat T: \Ac^* \longrightarrow \Ac^*,\; a_1\ldots a_n \longmapsto	T(a_1)\ldots T(a_n) \,. \]

Define the iterates $\tau^{(n)} = {\hat T}^n(0)$.
And consider the bitwise flip
\[ \widetilde{a} = 1-a \,.\]
Note that $\hat T$ and the bitwise flip commute.
%and note that $|\tau^{(n)}| = 2^n$ and
%\[
%\tau^{(n+1)} = \tau^{(n)}\,\tau^{(n)}
%\quad\text{and}\quad
%\bigl(\tau^{(n)}\bigr)_{2^n} \equiv n \pmod2.
%\]

\begin{lemma}[Recursive structure]
	\label{lem:tm-recursion}
	For all $n \geqslant 0$, one has
	\[
	\tau^{(n+1)} = \tau^{(n)}\,\widetilde{\tau^{(n)}},
	\qquad |\tau^{(n)}|=2^n,
	\qquad (\tau^{(n)})_{2^n}\equiv n\pmod2.
	\]
\end{lemma}
\begin{proof}
	We only show $\tau^{(n+1)} = \tau^{(n)}\,\widetilde{\tau^{(n)}}$ as the other two statements are direct corollaries.
	The base case is $\tau^{(0)}=0$, $\tau^{(1)}=01=\tau^{(0)}\,\widetilde{\tau^{(0)}}$.
	Now assume that 
	$\tau^{(n+1)} = \tau^{(n)}\,\widetilde{\tau^{(n)}}$
	then
	\begin{align*}
		\tau^{(n+2)} &= \hat T\left(\tau^{(n+1)}\right)\\
		&= \hat T\left( \tau^{(n)}\,\widetilde{\tau^{(n)}}\right)\\
		&= \hat T\left( \tau^{(n)} \right) \hat T\left(\widetilde{\tau^{(n)}}\right)\\
		&= \hat T\left( \tau^{(n)} \right) \widetilde{\hat T(\tau^{(n)})}\\
		&= \tau^{(n+1)} \widetilde{\tau^{(n+1)}} \qedhere
	\end{align*}
\end{proof}

Let $\tau=\lim_{n\to\infty}\tau^{(n)}$ and define the shift space
\[
X_\tau = \overline{\{\sigma^k \tau\mid k\in\Z\}}\subset\{0,1\}^\Z,
\]
where $\sigma$ is the left shift.
One easily checks that $X_\tau$ is minimal and uniquely ergodic:
\begin{lemma}[Strict ergodicity]\label{lem:tm-ergodic}
The subshift $(X_\tau,\sigma)$ is minimal and uniquely ergodic.
\end{lemma}
\begin{proof}
	Minimality is equivalent to almost periodicity.
	By the follows from the recursive structure in Lemma~\ref{lem:tm-recursion} $\tau$ is indeed almost periodic.

	Here are the details.
	Let $w$ appear anywhere in $\tau$.
	Then there is $n \in \N$ such that $w \in \tau^{(n)}$.
	Furthermore we know that $\widetilde{w}$ appears in $\widetilde{\tau^{(n)}}$.
	Hence, both $w$ and $\widetilde{w}$ appear in $\tau^{(n+1)}$.
	Thus, also both $w$ and $\widetilde{w}$ appear in $\widetilde{\tau^{(n+1)}}$.
	Therefore, $w$ appears in both $\tau^{(n+1)}$ and $\widetilde{\tau^{(n+1)}}$.
	Now the recursive structure shows that
	\[ \tau = F^{\tau_1}(\tau^{(n+1)})F^{\tau_2}(\tau^{(n+1)})F^{\tau_3}(\tau^{(n+1)}) \ldots = \bigoplus_{k \in \N} F^{\tau_1}(\tau^{(n+1)} ) \,,\]
	where $F(v) = \widetilde{v}$ and thus $F^1(v) = F(v) = \widetilde{v}$ and $F^0(v) = \Id(v) = v$.
	This shows that $w$ appears in every block of lenght $2^{n+2}$ --- as each such block contains one copy of $\tau^{(n+1)}$ or $\widetilde{\tau^{(n+1)}}$ completely.
	So $\tau$ is almost periodic.
	In order to show the minimality we must show that every orbit is dense.
	This means that every infinite word contains the same finite words.
	Let $\omega \in X_\tau = \overline{\{ \sigma^n(\tau) \; \vert \; n \in \N \}}$.
	Then clearly every finite word occuring in $\omega$ must occur in $\tau$,
	as $\omega$ is the limit of bigger and bigger finite subwords of $\tau$.
	Let now $w$ be a finite subword of $\tau$.
	Then there is $N \in \N$ such that $w$ appears in every subword of $\tau$ of lenght $N$.
	As $\omega$ is in the orbit closure of $\tau$ there is $n \in \N$ such that
	$\omega_i = \tau_{n+i}$ for $i \in \{ 1,\ldots,N\}$.
	So $w$ appears in $\omega$. 

	Unique ergodicity is equivalent to uniform convergence of ergodic averages for a dense set of continuous functions.
	Let $w \in \Ac^*$ be any finite word with lenght $n$.
	The cylinder set $[w]$ associated to $w$ is given by
	\[ [w] := \left\{ \omega \in X_\tau \; \vert \; \forall i \in \{ 1,\ldots,n\}: \omega_i = w_i \} \right\} \]
	As the set of cylinder sets forms a base of the topology consisting of open and closed sets we learn that the set of its indicator functions is dense in the space of all continuous functions.
	For $k\in\N$ let
	\[
	c_k\;:=\;\#_{w}\!\bigl(\tau^{(k)}\bigr)
	\quad\text{and}\quad
	\widetilde c_k\;:=\;\#_{\widetilde w}\!\bigl(\tau^{(k)}\bigr),
	\]
	where $\#_{u}(v)$ denotes the number of (possibly overlapping)
	occurrences of a word $u$ inside $v$ and $\widetilde{w}=F(w)$ is the
	reflection defined in Lemma~\ref{lem:tm-recursion}.
	Note that 
	\[ c_k = \sum_{i=1}^{2^k} \1_{[w]}(\sigma^i(\tau)) \,.\]
	By the recursive structure
	\[
	\tau^{(k+1)}\;=\;\tau^{(k)}\,\widetilde{\tau^{(k)}},
	\]
	we obtain
	\[
	c_{k+1}\;=\;c_{k}\;+\;\widetilde c_{k} + E \,,
	\tag{$\ast$}
	\]
	where $E$ is at most the lenght of $w$.
	Because $F$ is an involution we also have, $\widetilde c_{k+1} = c_{k}+\widetilde c_{k} + E$.
	The reason for this additional summand is that copies of $w$ could appear around the position where we glue $\tau^{(n)}$ and $\widetilde{\tau^{(n)}}$ together which we have not yet counted.
	well.
	Let $N \in \N$ be large enough such that $w$ appears in both $\tau^{(N)}$ and $\widetilde{\tau^{(N)}}$.
	Then $|c^{(N+k)} - \widetilde{c^{(N+k)}}| \leqslant n$ for any $k \in \N$ and thus
	\[ 2^k c^{(N)} - 2n\cdot k \leqslant c^{(N+k)} \leqslant 2^k c^{(N)} + 2n\cdot k \,. \]
    Therefore,	
	\begin{align*}
		\lim_{k \to \infty} \frac{c^{(k)}}{2^k} &= \lim_{k \to \infty} \frac{c^{(N+k)}}{2^{N+k}} \\
		&= \lim_{k \to \infty} \frac{c^{(N)}}{2^{N}} \\
	\end{align*}
	Dividing by $2^{k+1}$ and noting that $|\tau^{(k)}|=2^{k}$, we get
	\[
	\Bigl|\frac{c_{k+1}}{2^{k+1}}
		  -\frac{c_{k}}{2^{k}}\Bigr|
	\;\le\;\frac{n-1}{2^{k+1}}
	\;\xrightarrow[k\to\infty]{}0.
	\]
	Hence the sequence $\bigl(c_{k}/2^{k}\bigr)_{k}$ is Cauchy and therefore
	convergent.
	Write $\mu([w])$ for its limit.
	This value will turn out to be the measure of $[w]$ for the unique
	invariant measure $\mu$.

	\medskip
	Now let $\omega\in X_\tau$ and choose integers $m_j\to\infty$ with
	\[
	\omega\;=\;\lim_{j\to\infty}\sigma^{m_j}\tau
	\quad\text{in }X_\tau.
	\]
	For $N>n$ pick $k$ such that $2^{k}\le N<2^{k+1}$.
	Partition the first $N$ symbols of $\omega$ into consecutive blocks of
	length $2^{k}$ (the last block may be shorter).
	Each such block is either $\tau^{(k)}$ or $\widetilde{\tau^{(k)}}$ up to at
	most $n-1$ boundary symbols, because every length-$2^{k}$ window in
	$\tau$ is of that form and the same holds for any limit point
	$\omega$.
	A direct counting argument using $(\ast)$ yields
	\[
	\Bigl|A_N(\mathbf 1_{[w]},\omega)-\frac{c_k}{2^{k}}\Bigr|
	\;\le\;\frac{2n}{2^{k}}
	\;\le\;\frac{2n}{N}.
	\]
	Taking $N\to\infty$ we conclude
	\[
	\lim_{N\to\infty}A_N(\mathbf 1_{[w]},\omega)\;=\;\mu([w]),
	\qquad
	\text{uniformly in }\omega\in X_\tau.
	\]

	\medskip
	Uniform convergence for the indicators of all cylinder sets extends,
	by density and a standard $\varepsilon/3$ argument, to every
	$f\in C(X_\tau)$.
	Therefore the limiting functional $\int f\,d\mu$ defines a
	\emph{unique} $\sigma$-invariant Borel probability measure $\mu$ on
	$X_\tau$.
	Combining unique ergodicity with minimality proved above, we have
	shown that $(X_\tau,\sigma)$ is strictly ergodic.
	\qedhere
\end{proof}

\subsection{The Period Doubling Substitution}
Define another constant‑length substitution
\[
P:\ 0 \mapsto 01,\quad 1 \mapsto 00,
\]
and set $\gamma^{(n)}=P^n(0)$.  One shows by induction:
\begin{lemma}[Period doubling recursion]\label{lem:pd-recursion}
For all $n\ge1$,
\[
\gamma^{(n+1)} = \gamma^{(n)}\,\gamma^{(n-1)}\,\gamma^{(n-1)},
\qquad |\gamma^{(n)}|=2^n,
\qquad (\gamma^{(n)})_{2^n}\equiv n\pmod2.
\]
\end{lemma}
\begin{proof}
Direct induction using the definition of $P$.  (See Lemma 3.13 in Jäger 2025.)
\end{proof}

Let $\gamma=\lim_{n\to\infty}\gamma^{(n)}$ in $\{0,1\}^\N$ and extend to the two‑sided flow
\[
X_\gamma=\overline{\{\sigma^k\gamma\mid k\in\Z\}},
\]
called the \emph{period doubling subshift}.  One shows that $\gamma$ is a regular one‑sided
Toeplitz sequence and hence $(X_\gamma,\sigma)$ is a regular Toeplitz flow whose maximal equi\-con\-ti\-nuous factor is the dyadic odometer~$\Omega$.  The factor map
\[
p:\ X_\gamma\to\Omega
\quad\text{is uniquely determined by requiring }p(\gamma)=0.
\]
This completes the construction of the period doubling system.  (See Lemma 3.14 in Jäger 2025.)

\subsection{Factor Map from Thue–Morse to Period Doubling}
Define
\[
h:\ X_\tau\to X_\gamma,
\qquad
h(x)_n = \begin{cases}0,&x_{n+1}\neq x_n,\\1,&x_{n+1}=x_n.\end{cases}
\]
\begin{theorem}[Two‑to‑one factor]\label{thm:tm-to-pd}
The map $h$ is a continuous onto factor map which satisfies
\[
h\circ\sigma = \sigma\circ h,
\]
every fiber $h^{-1}(y)$ has cardinality exactly $2$, and thus $(X_\tau,\sigma)$ is a 2‑fold extension of $(X_\gamma,\sigma)$.
\end{theorem}
\begin{proof}
One checks at the level of finite approximants that $h(\tau^{(n)})=\gamma^{(n-1)}$ and hence by continuity $h(\tau)=\gamma$.  Commutation with $\sigma$ is immediate from the definition of $h$, minimality of both systems gives surjectivity, and injectivity fails only in identifying each sequence with its bit‑complement reflection.  (Cf. Theorem 4.4 in Jäger 2025.)
\end{proof}

\subsection{Dyadic Odometer Extension and Almost‑Surely One‑to‑One Property}
We now consider the factor
\[
\pi:\ X_\gamma\to\Omega,
\]
to the dyadic odometer defined above.  Since $\gamma$ is regular, Theorem 3.7 in Jäger 2025 shows that $\pi$ is almost‑surely one‑to‑one under the Haar measure on $\Omega$.  Moreover, one checks from the Toeplitz construction that each fiber has at most two points, corresponding to the two ways of filling the unique "hole" at each stage.  Hence:
\begin{proposition}
The period doubling subshift $(X_\gamma,\sigma)$ is an almost‑surely (Haar‑measure) one‑to‑one, and at most two‑to‑one, extension of the dyadic odometer.
\end{proposition}
\begin{proof}
Regularity of $\gamma$ gives $\pi$ is $m$‑a.s.\;injective, while the Toeplitz skeleton has exactly one hole per level, so topologically each orbit in $\Omega$ has at most two preimages in $X_\gamma$.  (See Theorems 3.7 and 3.11, and Remark 3.15 in Jäger 2025.)
\end{proof}
