\section{Example: Thue Morse Subshift}

We consider the finite alphabet $\Ac := \{ 0, 1 \}$.
The set of finite words is given as
\[ \Ac^* := \bigcup_{n\in\N} \Ac^n \]
and the infinite sequences are given by $\Ac^\N$.
Consider the following substitution
\[ T: \Ac \longrightarrow \Ac^*,\; a \longmapsto	\begin{cases} 01 & a=0 \\ 10 & a=1 \end{cases} \,.\]
This of course extends to a substitution
\[ \hat T: \Ac^* \longrightarrow \Ac^*,\; a_1\ldots a_n \longmapsto	T(a_1)\ldots T(a_n) \,. \]

Define the iterates $\tau^{(n)} = {\hat T}^n(0)$.
And consider the bitwise flip
\[ \tilde{a} = 1-a \,.\]
Note that $\hat T$ and the bitwise flip commute.
%and note that $|\tau^{(n)}| = 2^n$ and
%\[
%\tau^{(n+1)} = \tau^{(n)}\,\tau^{(n)}
%\quad\text{and}\quad
%\bigl(\tau^{(n)}\bigr)_{2^n} \equiv n \pmod2.
%\]

\begin{lemma}[Recursive structure]
	\label{lem:tm-recursion}
	For all $n \geqslant 0$, one has
	\[
	\tau^{(n+1)} = \tau^{(n)}\,\tilde{\tau^{(n)}},
	\qquad |\tau^{(n)}|=2^n,
	\qquad (\tau^{(n)})_{2^n}\equiv n\pmod2.
	\]
\end{lemma}
\begin{proof}
	We only show $\tau^{(n+1)} = \tau^{(n)}\,\tilde{\tau^{(n)}}$ as the other two statements are direct corollaries.
	The base case is $\tau^{(0)}=0$, $\tau^{(1)}=01=\tau^{(0)}\,\tilde{\tau^{(0)}}$.
	Now assume that 
	$\tau^{(n+1)} = \tau^{(n)}\,\tilde{\tau^{(n)}}$
	then
	\begin{align*}
		\tau^{(n+2)} &= \hat T\left(\tau^{(n+1)}\right)\\
		&= \hat T\left( \tau^{(n)}\,\tilde{\tau^{(n)}}\right)\\
		&= \hat T\left( \tau^{(n)} \right) \hat T\left(\tilde{\tau^{(n)}}\right)\\
		&= \hat T\left( \tau^{(n)} \right) \tilde{\hat T(\tau^{(n)})}\\
		&= \tau^{(n+1)} \tilde{\tau^{(n+1)}} \qedhere
	\end{align*}
\end{proof}

Let $\tau=\lim_{n\to\infty}\tau^{(n)}$ and define the shift space
\[
X_\tau = \tilde{\{\sigma^k \tau\mid k\in\Z\}}\subset\{0,1\}^\Z,
\]
where $\sigma$ is the left shift.  One easily checks that $X_\tau$ is minimal and uniquely ergodic:
\begin{lemma}[Strict ergodicity]\label{lem:tm-ergodic}
The subshift $(X_\tau,\sigma)$ is minimal and uniquely ergodic.
\end{lemma}
\begin{proof}
Almost periodicity follows from the recursive structure in Lemma~\ref{lem:tm-recursion}, giving uniform return times for every finite subword; unique ergodicity then follows by the standard Perron–Frobenius argument for constant‑length substitutions.  (Cf. Lemma 4.2 in Jäger 2025.)
\end{proof}

\subsection{The Period Doubling Substitution}
Define another constant‑length substitution
\[
P:\ 0 \mapsto 01,\quad 1 \mapsto 00,
\]
and set $\gamma^{(n)}=P^n(0)$.  One shows by induction:
\begin{lemma}[Period doubling recursion]\label{lem:pd-recursion}
For all $n\ge1$,
\[
\gamma^{(n+1)} = \gamma^{(n)}\,\gamma^{(n-1)}\,\gamma^{(n-1)},
\qquad |\gamma^{(n)}|=2^n,
\qquad (\gamma^{(n)})_{2^n}\equiv n\pmod2.
\]
\end{lemma}
\begin{proof}
Direct induction using the definition of $P$.  (See Lemma 3.13 in Jäger 2025.)
\end{proof}

Let $\gamma=\lim_{n\to\infty}\gamma^{(n)}$ in $\{0,1\}^\N$ and extend to the two‑sided flow
\[
X_\gamma=\overline{\{\sigma^k\gamma\mid k\in\Z\}},
\]
called the \emph{period doubling subshift}.  One shows that $\gamma$ is a regular one‑sided
Toeplitz sequence and hence $(X_\gamma,\sigma)$ is a regular Toeplitz flow whose maximal equi\-con\-ti\-nuous factor is the dyadic odometer~$\Omega$.  The factor map
\[
p:\ X_\gamma\to\Omega
\quad\text{is uniquely determined by requiring }p(\gamma)=0.
\]
This completes the construction of the period doubling system.  (See Lemma 3.14 in Jäger 2025.)

\subsection{Factor Map from Thue–Morse to Period Doubling}
Define
\[
h:\ X_\tau\to X_\gamma,
\qquad
h(x)_n = \begin{cases}0,&x_{n+1}\neq x_n,\\1,&x_{n+1}=x_n.\end{cases}
\]
\begin{theorem}[Two‑to‑one factor]\label{thm:tm-to-pd}
The map $h$ is a continuous onto factor map which satisfies
\[
h\circ\sigma = \sigma\circ h,
\]
every fiber $h^{-1}(y)$ has cardinality exactly $2$, and thus $(X_\tau,\sigma)$ is a 2‑fold extension of $(X_\gamma,\sigma)$.
\end{theorem}
\begin{proof}
One checks at the level of finite approximants that $h(\tau^{(n)})=\gamma^{(n-1)}$ and hence by continuity $h(\tau)=\gamma$.  Commutation with $\sigma$ is immediate from the definition of $h$, minimality of both systems gives surjectivity, and injectivity fails only in identifying each sequence with its bit‑complement reflection.  (Cf. Theorem 4.4 in Jäger 2025.)
\end{proof}

\subsection{Dyadic Odometer Extension and Almost‑Surely One‑to‑One Property}
We now consider the factor
\[
\pi:\ X_\gamma\to\Omega,
\]
to the dyadic odometer defined above.  Since $\gamma$ is regular, Theorem 3.7 in Jäger 2025 shows that $\pi$ is almost‑surely one‑to‑one under the Haar measure on $\Omega$.  Moreover, one checks from the Toeplitz construction that each fiber has at most two points, corresponding to the two ways of filling the unique "hole" at each stage.  Hence:
\begin{proposition}
The period doubling subshift $(X_\gamma,\sigma)$ is an almost‑surely (Haar‑measure) one‑to‑one, and at most two‑to‑one, extension of the dyadic odometer.
\end{proposition}
\begin{proof}
Regularity of $\gamma$ gives $\pi$ is $m$‑a.s.\;injective, while the Toeplitz skeleton has exactly one hole per level, so topologically each orbit in $\Omega$ has at most two preimages in $X_\gamma$.  (See Theorems 3.7 and 3.11, and Remark 3.15 in Jäger 2025.)
\end{proof}
