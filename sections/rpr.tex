\section{Regionally Proximal Relation}

\begin{definition}
  \label{def:rpr}
  Let $(X, T)$ be a TDS.
  Two points $x,y \in X$ are called \emph{regionally proximal}
  if and only if for each $\varepsilon > 0$ there exist $x' \in B_\varepsilon(x), y' \in B_\varepsilon(y)$
  and $t \in T$ such that $d(tx', ty') < \varepsilon$.
  We denote the set of regionally proximal pairs by $Q_2(X) \subseteq X \times X$.
\end{definition}


\begin{theorem}
  \label{rprchar}
  Let $(X, T)$ be a TDS.
  The set of regionally proximal pairs is exactly
  \[ Q_2(X) = \bigcap_{U \in \mathcal{U}(\Delta(X))} \overline{TU}  \]
  where $\Delta(X) \subset X \times X$ is the diagonal of the product space.
\end{theorem}

\begin{proof}
  For the first step, we see that \cref{def:rpr} is equivalent to the following characterization:
  The pair $(x, y) \in X \times X$ is regionally proximal if and only if there are sequences $(x_n)_n, (y_n)_n$ in $X$ and $(t_n)_n$ in $T$
  such that
  \begin{equation}
    \label{eq:rprchar}
    x_n \to x, \ y_n \to y \ \text{and} \ d(t_n x_n, t_n y_n) \to 0.
  \end{equation}
  We can rewrite this statement to say
  \[ (x_n, y_n) \to (x, y) \ \text{and} \ t_n(x_n, y_n) \in B_\varepsilon(\Delta(X)) \]
  for each $\varepsilon > 0$ and using that $X$ is compact this is equivalent to for each $U \in \mathcal{U}(\Delta(X))$
  \[ (x_n, y_n) \to (x, y) \ \text{and} \ t_n(x_n, y_n) \in U \]
  and it follows
  \[ (x_n, y_n) \to (x, y) \ \text{and} \ (x_n, y_n) \in TU \]
  and therefore $(x, y) \in \overline{TU}$.
  Since $U$ was arbitrary this shows the first direction of the equivalence.

  Since $X$ is metrizable we can choose a countable system of neighborhoods $B_n$ of $\Delta(X)$.
  We can follow from the characterization in \cref{rprchar} that if $(x, y) \in Q_2(X)$
  we have that for each $U \in \mathcal{U}(\Delta(X))$ and in particular $B_n \in \mathcal{U}(\Delta(X))$
  there is $(x_n, y_n) \to (x, y)$ such that $(x_n, y_n) \in TU$ and thus $t_n(x_n, y_n) \in U$.
  The diagonal sequence over all neighborhoods $B_n$ ordered by inclusion satisfies \ref{eq:rprchar}.
\end{proof}

\begin{theorem}
  The regionally proximal equation is closed, invariant, symmetric and reflexive.
\end{theorem}

\begin{proof}
  Let $(X, T)$ be a TDS.
  The set $Q_2(X)$ is an intersection of closed sets and therefore closed.
  It is reflexive, since $\Delta(X) \subseteq \overline{TU}$ for each $U \in \mathcal{U}(\Delta(X))$.
  The relation is symmetric since the definition is symmetric with respect to $x$ and $y$.
  For invariance, let $(x, y) \in Q_2(X)$ and $t \in T$.
  Choose sequences as defined in \cref{eq:rprchar}.
  Then for $(tx_n)_n$, $(t y_n)_n$ and $(t_nt^{-1})_n$ we have $t x_n \to tx$ and $t y_n \to ty$ with $d(t_nt^{-1}x_n, t_nt^{-1}y_n) = d(x_n, y_n) \to 0$.
\end{proof}

Note that in general the regionally proximal relation is not transitive.
The next construction is a counterexample.

\begin{example}
  Consider the compact set $X = [0, 2]$ and the group $\mathbb{R}$ under addition.
  The real numbers act on $[0, 2]$ by the group actions defined by
  \begin{equation*}
    \mathbb{R} \times X \to X, \ (t, x) \mapsto 
    tx := \begin{cases}
      0 & x = 0 \\
      \frac{1}{1 + (\frac{1}{x} - 1)e^{-t}} & x \in (0, 1] \\
      \frac{1}{1 + (\frac{1}{x - 1} - 1)e^{-t}} & x \in (1, 2]
    \end{cases}
  \end{equation*}
  One can check that $0x = x$ and $t(sx) = (t + s)x$ for all $r, s \in \mathbb{R}$ and $x \in X$.
  Also the singeltons $\{ 0 \}, \{ 1 \}$ and $\{ 2 \}$ are closed invariant subspaces.
  So the TDS is not minimal.

  {\color{red} Hier ein Bild}
  It is easy to see that $0$ and $1$ are a regionally proximal pair,
  since one can choose a point arbitrarily larger than $0$ and smaller than $1$.
  Both will tend to $1$ but never reach $1$ as $t$ gets bigger,
  so their distance will get arbitrarily small.
  The same is true for the points $1$ and $2$.

  At the same time $0$ and $2$ are not regionally proximal,
  since a point in a (small enough) neighborhood will tend to $1$ and a point in a neighborhood of $2$ will tend to $2$ as $t$ gets bigger
  (and similarly if $t$ gets smaller).
\end{example}

\begin{theorem}
  Let $(X, T)$ be an equicontinuous TDS.
  Then $Q_2(X) = \Delta(X)$.
\end{theorem}

\begin{proof}
  Let $d$ be an invariant metric.
  Let $x \neq y \in X$ with a distance $D := d(x, y) > 0$.
  Choose $\varepsilon = \frac{D}{4}$.
  For each $x' \in B_\varepsilon(x)$, $y' \in B_\varepsilon(y)$ and $t \in T$ we have
  \[ d(tx', ty') = d(x', y') \geq | d(x', y) - d(y, y') | \geq | | d(x, x') - d(x, y) | - d(y, y') | = D - d(x, x') - d(y, y') \geq \frac{D}{2} > \varepsilon. \]
\end{proof}

\begin{theorem}
  Let $(X, T)$ be a minimal TDS with an invariant Borel measure
  and $R$ be the ICER associated with the MEF of $X$.
  Then $Q_2(X) = R$.
\end{theorem}
