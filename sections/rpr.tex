\section{Regionally Proximal Relation}

\begin{definition}
  \label{def:rpr}
  Let $(X, T)$ be a TDS.
  Two points $x,y \in X$ are called \emph{regionally proximal}
  if and only if for each $\varepsilon > 0$ there exist $x' \in B_\varepsilon(x), y' \in B_\varepsilon(y)$
  and $t \in T$ such that $d(tx', ty') < \varepsilon$.
  We denote the set of regionally proximal pairs by $Q_2(X) \subseteq X \times X$.
\end{definition}


\begin{theorem}
  \label{rprchar}
  Let $(X, T)$ be a TDS.
  The set of regionally proximal pairs is exactly
  \[ Q_2(X) = \bigcap_{U \in \mathcal{U}(\Delta(X))} \overline{TU}  \]
  where $\Delta(X) \subset X \times X$ is the diagonal of the product space.
\end{theorem}

\begin{proof}
  For the first step, we see that \cref{def:rpr} is equivalent to the following characterization:
  The pair $(x, y) \in X \times X$ is regionally proximal if and only if there are sequences $(x_n)_n, (y_n)_n$ in $X$ and $(t_n)_n$ in $T$
  such that
  \begin{equation}
    \label{eq:rprchar}
    x_n \to x, \ y_n \to y \ \text{and} \ d(t_n x_n, t_n y_n) \to 0.
  \end{equation}
  We can rewrite this statement to say
  \[ (x_n, y_n) \to (x, y) \ \text{and} \ t_n(x_n, y_n) \in B_\varepsilon(\Delta(X)) \]
  for each $\varepsilon > 0$ and using that $X$ is compact this is equivalent to for each $U \in \mathcal{U}(\Delta(X))$
  \[ (x_n, y_n) \to (x, y) \ \text{and} \ t_n(x_n, y_n) \in U \]
  and it follows
  \[ (x_n, y_n) \to (x, y) \ \text{and} \ (x_n, y_n) \in TU \]
  and therefore $(x, y) \in \overline{TU}$.
  Since $U$ was arbitrary this shows the first direction of the equivalence.

  Since $X$ is metrizable we can choose a countable system of neighborhoods $B_n$ of $\Delta(X)$.
  We can follow from the characterization in \cref{rprchar} that if $(x, y) \in Q_2(X)$
  we have that for each $U \in \mathcal{U}(\Delta(X))$ and in particular $B_n \in \mathcal{U}(\Delta(X))$
  there is $(x_n, y_n) \to (x, y)$ such that $(x_n, y_n) \in TU$ and thus $t_n(x_n, y_n) \in U$.
  The diagonal sequence over all neighborhoods $B_n$ ordered by inclusion satisfies \ref{eq:rprchar}.
\end{proof}

\begin{theorem}
  The regionally proximal equation is closed, invariant, symmetric and reflexive.
\end{theorem}

\begin{proof}
  Let $(X, T)$ be a TDS.
  The set $Q_2(X)$ is an intersection of closed sets and therefore closed.
  It is reflexive, since $\Delta(X) \subseteq \overline{TU}$ for each $U \in \mathcal{U}(\Delta(X))$.
  The relation is symmetric since the definition is symmetric with respect to $x$ and $y$.
  For invariance, let $(x, y) \in Q_2(X)$ and $t \in T$.
  Choose sequences as defined in \cref{eq:rprchar}.
  Then for $(tx_n)_n$, $(t y_n)_n$ and $(t_nt^{-1})_n$ we have $t x_n \to tx$ and $t y_n \to ty$ with $d(t_nt^{-1}x_n, t_nt^{-1}y_n) = d(x_n, y_n) \to 0$.
\end{proof}

Note that in general the regionally proximal relation is not transitive.
The next construction is a counterexample.

\begin{example}
  Consider the compact set $X = [0, 2]$ and the group $\mathbb{R}$ under addition.
  The real numbers act on $[0, 2]$ by the group actions defined by
  \begin{equation*}
    \mathbb{R} \times X \to X, \ (t, x) \mapsto 
    tx := \begin{cases}
      0 & x = 0 \\
      \frac{1}{1 + (\frac{1}{x} - 1)e^{-t}} & x \in (0, 1] \\
      \frac{1}{1 + (\frac{1}{x - 1} - 1)e^{-t}} & x \in (1, 2]
    \end{cases}
  \end{equation*}
  One can check that $0x = x$ and $t(sx) = (t + s)x$ for all $r, s \in \mathbb{R}$ and $x \in X$.
  Also the singeltons $\{ 0 \}, \{ 1 \}$ and $\{ 2 \}$ are closed invariant subspaces.
  So the TDS is not minimal.

  {\color{red} Hier ein Bild}
  It is easy to see that $0$ and $1$ are a regionally proximal pair,
  since one can choose a point arbitrarily larger than $0$ and smaller than $1$.
  Both will tend to $1$ but never reach $1$ as $t$ gets bigger,
  so their distance will get arbitrarily small.
  The same is true for the points $1$ and $2$.

  At the same time $0$ and $2$ are not regionally proximal,
  since a point in a (small enough) neighborhood will tend to $1$ and a point in a neighborhood of $2$ will tend to $2$ as $t$ gets bigger
  (and similarly if $t$ gets smaller).
\end{example}
\begin{proposition}
  \label{prop:phiSqQ2XcQ2Y}
  Let $\varphi : (X,T) \to (Y,T)$ a morphism of TDS. Then $(\varphi \times \varphi) (Q_2 (X)) \subset Q_2(Y)$.
\end{proposition}
\begin{proof}
  Let $(x_1, x_2) \in Q_2(X)$ and $\varepsilon \in (0,\infty)$.
  Since $X$ is compact $\varphi$ is uniformly continuous.
  Let $\delta \in (0, \infty)$ such that $\forall x_1^\prime, x_2^\prime \in X:$ $d(x_1^\prime, x_2^\prime)<\delta \Rightarrow d(\varphi (x_1^\prime), \varphi(x_2^\prime)) < \varepsilon$.
  Let $\tilde{\varepsilon} := \min \{\delta, \varepsilon\}$.
  Let $z_1 \in B_{\tilde{\varepsilon}} (x_1), z_2 \in  B_{\tilde{\varepsilon}}(x_2)$ and $t \in T$ such that $d(t z_1, tz_2) < \tilde{\varepsilon} \leq \delta$ (definition of $Q_2(X)$).
  Then clearly $z_1 \in B_\varepsilon (x_1)$ and $z_2 \in B_\varepsilon  (x_2)$ as well.
  Furthermore
  \begin{equation*}
    d(t \varphi (z_1), t\varphi(z_2)) = d (\varphi ( t z_1) , \varphi (t z_2)) < \varepsilon.
  \end{equation*}
\end{proof}

\begin{theorem}
  Let $(X,T)$ a TDS and $\varphi : (X,T) \to (Y,T)$ an equicontinuous factor.
  Let $R_\varphi$ the ICER generated by $\varphi$. Then $Q_2(X) \subset R_\varphi$.
\end{theorem}
\begin{proof}
  By contradiction: Let $(x_1,x_2) \in Q_2(X)$ with $(x_1,x_2) \notin R_\varphi$.
  Let $y_1 := \varphi (x_1)$, $y_2 := \varphi (x_2)$.
  Then $y_1 \neq y_2$ by definition of $R_\varphi$ and $(y_1, y_2) \in Q_2 (Y)$ by \cref{prop:phiSqQ2XcQ2Y}.
  Let $d$ be a $T$-invariant metric on $Y$. Let $D := d(y_1, y_2) > 0$.
  Choose $\varepsilon = \frac{D}{4}$.
  For each $y_1' \in B_\varepsilon(y_1)$, $y_2' \in B_\varepsilon(y_2)$ and $t \in T$ we have using the triangle inequality:
  \begin{equation*}
    \begin{split}
      d(ty_1', ty_2') &= d(y_1', y_2') \\
      &\geq  d(y_1', y_2) - d(y_2, y_2')  \\
      &\geq d (y_1, y_2) - d(y_1, y_1')  - d(y_2, y_2')  \\
      &= D - d(y_1, y_1') - d(y_2, y_2') \geq \frac{D}{2} > \varepsilon.
    \end{split}
  \end{equation*}
  Contradicting the fact that $(y_1, y_2) \in Q_2(Y)$.
\end{proof}
\begin{corollary}
 Let $(X, T)$ be an equicontinuous TDS.
  Then $Q_2(X) = \Delta(X)$.
\end{corollary}
\begin{proof}
  $I : (X,T) \to (X,T)$ is an equicontinuous factor and $\Delta (X)$ the ICER generated by it.
  \end{proof}
\begin{corollary}
 Let $(X, T)$ be a TDS and $R$ the ICER generated by the MEF.
  Then $Q_2(X) \subset R$.
\end{corollary}
\begin{proof}
  The MEF is an equicontinuous factor.
  \end{proof}

\begin{theorem}
  Let $(X, T)$ be a minimal TDS with an invariant Borel probability measure
  and $R$ be the ICER associated with the MEF of $X$.
  Then $Q_2(X) = R$.
\end{theorem}

\begin{proof}
  See Auslander p.130 Thm. 8.
\end{proof}

\begin{example}
	We consider the one dimensional torus as $\mathbb{T} = \mathbb{R}/\mathbb{Z}$, where we write $[x]$ for the coset $x + \mathbb{Z}$ for $x \in \mathbb{R}$. The metric on $\mathbb{T}$ it is given by
	\begin{align*}
		d_\mathbb{T}([x], [y]) := \min_{n \in \mathbb{Z}} |x - y + n|
	\end{align*}
	for all $[x], [y] \in \mathbb{T}$ and the on $\mathbb{T}^2$ by
	\begin{align*}
		d_{\mathbb{T}^2}(([x_1], [y_1]), ([x_2], [y_1])) := d_\mathbb{T}([x_1], [x_2]) + d_\mathbb{T}([y_1], [y_2])
	\end{align*}
	for all $([x_1], [y_1]), ([x_2], [y_2]) \in \mathbb{T}^2$.
	
	We look the $\mathbb{Z}$-action arising from to the skew-rotation $\alpha: \mathbb{T}^2 \to \mathbb{T}^2$ defined by $\alpha([x], [y]) = ([x + a], [y + x])$ for all $x, y \in \mathbb{T}$, with a fixed $a \in \mathbb{R}$. Then
	\begin{align*}
		Q_2(\mathbb{T}^2) = \{(([x_1], [y_1]), ([x_2], [y_2])) \in \mathbb{T}^2 \times \mathbb{T}^2;\ [x_1]=[x_2]\}.
	\end{align*}
\end{example}
\begin{proof}
	Fix $[x], [y_1], [y_2] \in \mathbb{T}$ and $\varepsilon > 0$. Without loss of generality assume $y_1, y_2 \in [0, 1)$ and $y_1 \leq y_2$. Then there are $0 \leq b < \frac{\varepsilon}{2}$ and $n \in \mathbb{N}_0$ such that
	\begin{align*}
		0 < y_2 - y_1 - \frac{n(n-1)}{2}b < \frac{\varepsilon}{2}.
	\end{align*}
	Then $([b+x], [y_1]) \in \operatorname{B}_\varepsilon[x], [y_1])$ and $([x],[y_2]) \in \operatorname{B}_\varepsilon([x], [y_2])$ and we have
	\begin{align*}
		&d_{\mathbb{T}^2}(n([b+x], [y_1]), n([x], [y_2]))\\
		= &d_{\mathbb{T}^2}(([na+b+x], [\frac{n(n-1)}{2}a + n(b+x) + y_1]), ([na+x], [\frac{n(n-1)}{2}a + nx + y_2]))\\
		= &d_\mathbb{T}([na+b+x], [na+x]) + d_\mathbb{T}([\frac{n(n-1)}{2}a + n(b+x) + y_1], [\frac{n(n-1)}{2}a + nx + y_2])\\
		\leq &b + y_2 - y_1 - \frac{n(n-1)}{2}b\\
		< &\frac{\varepsilon}{2} + \frac{\varepsilon}{2}\\
		= &\varepsilon,
	\end{align*}
	so $(([x], [y_1]), ([x], [y_2]) \in Q_2(\mathbb{T}^2)$, which proves the "$\supseteq$"-direction.
	
	Now fix $([x_1], [y_1]), ([x_2], [y_2]) \in \mathbb{T}^2$ and assume $x_1 \neq x_2$. Without loss of generality we can assume $x_1, x_2 \in [0, 1)$ and $x_1 < x_2$. Choose $0 < \varepsilon < \min\{\frac{x_2-x_1}{3}, \frac{x_1}{3}, \frac{1 - x_2}{3}\}$. Now let $([x_1'], [y_1']) \in \operatorname{B}_\varepsilon([x_1], [y_1])$, $([x_2'], [y_2']) \in \operatorname{B}_\varepsilon([x_2], [y_2])$ and $n \in \mathbb{N}_0$. Again we can assume without loss of generality that $x_1', x_2' \in [0, 1)$. By the choice of $\varepsilon$ we have
	\begin{align*}
		&d_{\mathbb{T}^2}(\pm n([x_1'], [y_1']), \pm n([x_2'], [y_2']))\\
		= &d_{\mathbb{T}^2}(([\pm na + x_1'], [\pm \frac{n(n-1)}{2} \pm nx_1' + y_1'], ([\pm \frac{n(n-1)}{2}a \pm nx_1' + y_1], [\pm \frac{n(n-1)}{2}a \pm nx_2' + y_2']))\\
		\geq &d_\mathbb{T}([\pm na + x_1'], [\pm na + x_2'])\\
		= &x_2' - x_1'\\
		> &\varepsilon,
	\end{align*}
	so $(([x_1], [y_1]), ([x_2], [y_1])) \notin Q_2(\mathbb{T}^2)$, which proves the "$\subseteq$"-direction.
\end{proof}
