\section{Thue Morse Subshift}

\begin{frame}
	Let $\sigma$ be the left shift on $\{0, 1\}^\mathbb{Z}$. Then $(\{0, 1\}^\mathbb{Z}, \sigma)$ is a TDS.\pause
	\begin{columns}
		\begin{column}{.5\textwidth}
			\begin{example}[Thue Morse Subshift]
				Take the substitution
				\begin{align*}
					\begin{dcases}
						0 \mapsto 01\\
						1 \mapsto 10
					\end{dcases}.
				\end{align*}\pause
				It has a fixed point $v \in \{0, 1\}^\mathbb{Z}$ and gives a minimal subshift
				\begin{align*}
					X := \overline{\{\sigma^nv;\ n \in \mathbb{Z}\}}.
				\end{align*}
			\end{example}\pause
		\end{column}
		\begin{column}{.5\textwidth}
			\begin{example}[Period Doubling Subshift]
				Take the substitution
				\begin{align*}
					\begin{dcases}
						0 \mapsto 01\\
						1 \mapsto 00
					\end{dcases}.
				\end{align*}\pause
				It has a fixed point $w \in \{0, 1\}^\mathbb{Z}$ and gives a minimal subshift
				\begin{align*}
					Y := \overline{\{\sigma^nw;\ n \in \mathbb{Z}\}}.
				\end{align*}
			\end{example}\pause
		\end{column}
	\end{columns}
	\medskip

	Then there is a $2$-to-$1$ factor $\pi: X \to Y$ defined by
	\begin{align*}
		\pi((a_n)_{n \in \mathbb{Z}}) :=
		\begin{dcases}
			0 &; a_n \neq a_{n+1}\\
			1 &; a_n = a_{n+1}
		\end{dcases}.
	\end{align*} 
\end{frame}

\begin{frame}
	\frametitle{Dyadic Odometer}
	The dyadic integers are given by
	\begin{align*}
		\mathbb{Z}_2 &:= \varprojlim_{k \to \infty} \mathbb{Z}/2^k\mathbb{Z}\\
		&\cong \{(a_k)_{k \in \mathbb{N}} \in \prod_{k \in \mathbb{N}} \{0, \dots, 2^k-1\};\ a_{k+1} \in \{a_k, a_k + 2^k\}\}.
	\end{align*}\pause
	Equipped with pointwise addition they form a compact abelian group.\pause
	
	Define $\eta := (1, 1, \dots) \in \mathbb{Z}_2$ and define $d: \mathbb{Z}_2 \to \mathbb{Z}_2$ by $d(a) := a + \eta$.\pause Then $(\mathbb{Z}_2, d)$ is a minimal equicontinuous TDS called the \textbf{dyadic odometer}.\pause
	\medskip
	
	There is a factor map $\rho: Y \to \mathbb{Z}_2$, which is at most $2$-to-$1$. In particular $\mathbb{Z}_2$ is the MEF of $(Y, T)$ and $(X, T)$.
\end{frame}

\begin{frame}
	Putting this together we have
	\medskip
	
	\begin{align*}
		\begin{array}{rcc}
			\text{Thue Morse subshift:} &\quad (X, \sigma) \quad &5\text{-equicontinuous}\\
			\pi: &\downarrow &\\
			\text{period doubling subshift:} &\quad (Y, \sigma) \quad &3\text{-equicontinuous}\\
			\rho: &\downarrow &\\
			\text{dyadic odometer:} &\quad (\mathbb{Z}_2, d) \quad &\text{(2-)equicontinuous.}
		\end{array}
	\end{align*}\pause
	\medskip

	In particular $(\mathbb{Z}_2, d)$ is the MEF of $(X, \sigma)$ and $(Y, \sigma)$. Also $(Y, \sigma)$ is a $3$-MEF of $(X, \sigma)$.\pause
	
	Note that however the MEF of $(\{0, 1\}^\mathbb{Z}, \sigma)$ is trivial.
\end{frame}