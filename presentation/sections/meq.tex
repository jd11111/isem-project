\section{m-Equicontinuity}

\begin{frame}
	\begin{definition}[$m$-Equicontinuity \cite{Garcia-Ramos2024}\footnote{The corresponding sensitivity notion appeared in \cite{Jincheng2005}]
	    Fix $m \in \mathbb{N}$. Then $(X, T)$ is \emph{$m$-equicontinuous} if for all $\varepsilon > 0$ there exists a $\delta > 0$, such that for all $U \subseteq X$ open with $\operatorname{diam}(U) < \delta$, $x_1, \dots, x_m \in U$ and $t \in T$ there are $i, j \in \{1, \dots, m\}$ with $i \neq j$ such that $d(tx_i, tx_j) < \varepsilon$.
    \end{definition}
    \pause
    \begin{proposition}
	    If $(X, T)$ is $m$-equicontinuous for some $m \in \mathbb{N}$ it is also $m'$-equicontinuous for all $m' > m$.
    \end{proposition}
    Proof is clear by definition.

    Note that the "normal" equicontinuity is $2$-equicontinuity.
\end{frame}

\begin{frame}
    \begin{definition}[$m$-MEF]
        Fix $m \in \mathbb{N}$ with $m \geq 2$. Assume that $(X, T)$ is a minimal TDS and let its MEF be given by $\pi_{\mathrm{MEF}}: X \to X_{\mathrm{MEF}}$.
        Let $R_{\pi_{\mathrm{MEF}}}$ be the ICER generated by $\pi_{\text{MEF}}$. Let $\pi: (X,T) \to (Y,T)$ be a factor and $R_\pi$ the ICER generated by $\pi$.
        Then $\pi$ is called a \emph{$m$-MEF} if
        \begin{equation*}
            \forall A \in X/R_{\pi_{\mathrm{MEF}}} \exists B_1, \dots, B_{m-1} \in X/R_\pi: 
            A = \bigcup_{i=1}^{m-1} B_i.
        \end{equation*}
    \end{definition}
    Note that the indices of the $B_i$ only go up to $m-1$!
    \pause
    As before, we have:
    \begin{remark}
        Any $m$-MEF is also an $m'$-MEF for all $m' > m$
        since the $B_i$ in the definition of an $m$-MEF need not be pairwise distinct.
    \end{remark}
\end{frame}

\begin{frame}
    Importantly, we have
    \begin{proposition}[Lemma 4.14 in \cite{Garcia-Ramos2024}]
	    Fix $m \in \mathbb{N}$ with $m \geq 2$. Let $\pi : (X,T) \to (Y, T)$ be an $m$-MEF. Then $(Y, T)$ is $m$-equicontinuous.
    \end{proposition}
\end{frame}

\begin{frame}{Generalizations}
    We can generalize
    \begin{enumerate}
        \item the regionally proximal relation to an $m$-regionally proximal relation $Q_m(X) \subset X^{m}$.
        \item the diagonal to 
        $\Delta^m(X) := \{ (x_1, \dots, x_m) \in X^m \ | \ \text{there are} \ i, j \in \{1, \dots, m\} \ \text{such that} \ x_i = x_j \}$.
    \end{enumerate}

\end{frame}

\begin{frame}
    And get similarly to before
    \begin{theorem}[Theorem 4.6 in \cite{Garcia-Ramos2024}]
	    \label{thm:m-equiRelationChar}
	    Let $(X, T)$ be a minimal TDS.
	    Then $(X, T)$ is $m$-equicontinuous if and only if $Q_m(X) \setminus \Delta^m(X) = \varnothing$ for $2 \leq m \in \mathbb{N}$.
    \end{theorem}
    \pause
    \begin{corollary}
        Let $(X, T)$ be a minimal TDS and $\pi : X \to X_\text{MEF}$ its MEF.
        Then $m$-equicontinuous if and only if $\# \pi^{-1}(\{ y \}) < m$ for all $y \in X_\text{MEF}$.
    \end{corollary} 
\end{frame}